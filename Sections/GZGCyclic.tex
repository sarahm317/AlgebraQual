\documentclass[../AlgebraQualSolutions.tex]{subfiles}

\begin{document}

\section{$G/Z(G)$ is Cyclic}

	\begin{prob}{F12.G2}{F12.G2.b}
	Define the center of a group.
	\begin{enumerate}[(a)]
	\item Prove that if the order of $G$ is $p^k$ for some prime $p$ then $G$ has nontrivial center.
	\item Suppose that $p$ and $q$ are distinct primes. Prove that a non-abelian group of order $pq$ has trivial center.
	\end{enumerate}
	\end{prob}
	
	The center of a group $G$ is the subgroup
		\[Z(G) = \{g \in G: gx = xg \textrm{ for any $x \in G$} \} \]
	
	\begin{proof}
		See \ref{prob:F12.G2.a}
	\end{proof}

	\begin{proof}
		Suppose that $G$ is a non-abelian group of order $pq$ with $p \neq q$ both prime. Because $G$ is non-abelian and the order of a subgroup of $G$ must divide $pq$, $Z(G)$ is of order 1, $q$, or $p$. If $Z(G)$ is of order $p$ then $|G/Z(G)| = q$. Since $q$ is prime, this means that $G/Z(G)$ is cyclic. This implies that $G$ is abelian, a contradiction (see \ref{prob:F21.G1}). Similarly, if $Z(G)$ were of order $q$, then $G/Z(G)$ would be cyclic which contradicts $G$ being non-abelian. Therefore $|Z(G)| = 1$ meaning that the center is trivial.
	\end{proof}

	\begin{prob}{S20}{S20.G1}
	Let $G$ be a group and $H$ a subgroup of $G$ contained in the center $Z(G)$ of $G$ such that $G/H$ is cyclic.
	\begin{enumerate}[(a)]
	\item Show that $G$ is abelian.
	\item Show that every group of order $p^2$ with $p$ a prime is abelian. \emph{It may be assumed that a $p$-group has nontrivial center.}
	\end{enumerate}
	\end{prob}
	
	\begin{proof}
	Suppose that $G/H = \langle gH \rangle$ for some $g \in G$. Let $a, b \in G$. By assumption, $a = g^kH$ and $b = g^mH$ for some $k,m \in \N$. Therefore, $a = g^kh_1$ and $b= g^m h_2$. Since elements in $H$ commute with every element in $G$ and powers of $G$ commute with one another,
		\[ab = g^kh_1g^mh_2 = g^mh_2g^k h_1 = ba \]
	proving that $G$ is abelian.
	\end{proof}
	
	\begin{proof}
	Suppose that $|G| = p^2$ with $p$ a prime. Because $G$ is a $p$-group, it has nontrivial center and thus $|Z(g)| > 1$. By Lagrange's Theorem, $|Z(G)| = p$ or $|Z(G)| = p^2$. If $|Z(G)| =p^2$, $Z(G) = G$. If $|Z(G)| = p$, $\left|G/Z(G) \right| = p$. That is, $G/Z(G)$ is cyclic and therefore $G$ is abelian.
	\end{proof}
	
	\begin{prob}{F21.G1}{F21.G1}
	Let $G$ be a finite group and let $Z(G)$ denote the center of $G$.
	\begin{enumerate}[(a)]
	\item Prove that if $G/Z(G)$ is cyclic then $G$ is abelian.
	\item Does there exist a finite group $H$ such that $|H/Z(H)| = 7$? What if $|H/Z(H)| = 6$?
	\end{enumerate}
	\end{prob}
	
	\begin{proof}
	This is a special case of \ref{prob:S20.G1}.
	\end{proof}
	
	\begin{solution}
	Suppose that $|H/Z(H)| = 7$. As this group is of prime order, it must be cyclic. Therefore, by the previous result, $H$ is abelian. That is, $H = Z(H)$ which would imply that $|H/Z(H)| = 1$, a contradiction.\\
	
	If $H = S_3$ then $Z(H)$ is trivial and $|H: Z(H)| = 6$.
	\end{solution}

\end{document}