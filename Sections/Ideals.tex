\documentclass[../AlgebraQualSolutions.tex]{subfiles}

\begin{document}

\section{Ideals}

	\begin{prob}{F19}{F19.RF1}
	Prove that the set $N$ of nilpotent elements of a commutative ring $R$ is an ideal of $R$ and that $R/N$ has no nilpotent elements.
	\end{prob}
	
	\begin{proof}
	Since $0^1 = 0$, $0 \in N$ and thus $N \neq \varnothing$. Suppose that $x,y \in N$ are nonzero with $x^n = y^m = 0$ for some $m,n > 1$. Since $R$ is commutative,
		\[(x+y)^{mn} = \sum_{k=0}^{mn} {mn \choose k} x^k y^{mn - k} \]
	via the Binomial Theorem. Without loss of generality, assume that $n \leq m$. Then whenever $m \leq k \leq mn$, $x^k = 0$. Whenever $0 \leq k \leq m$, $m(n - 1) \leq mn - k \leq mn$. This implies that $k \geq m(n-1) \geq n(n-1) \geq n$ and therefore $y^k = 0$. Therefore $(x+y)^{mn} = 0$ and so $x + y \in N$. For any $r \in R$, $(rx)^n = r^nx^n = r^n \cdot 0 = 0$ where the first equality follows from $R$ being commutative. Therefore $rx \in N$. As $N$ is closed under addition and multiplication by elements in $R$, $N$ is an ideal.\\
	
	Suppose that $R/N$ has some nonzero nilpotent element. That is there exists $r \in R - N$ and $m \geq 1$ such that $(r+N)^m = N$. This implies that $r^m \in N$. Choose $n \geq 1$  such that $(r^m)^n = 0$. But this means that $r^{mn} = 0$, contradicting that $r \not\in N$. Thus there are no nonzero nilpotent elements in $R/N$.
	\end{proof}

\begin{prob}{S11.RF2}{S11.RF2}
	Let $R$ be a commutative ring with 1. Show that an ideal $M$ is maximal if and only if for all $r \in R \sm M$ there exists an element $x \in R$ such that $1 - rx \in M$.
\end{prob}

\begin{proof}
	Suppose first that $M$ is a maximal ideal. Then $R/M$ is a field. Let $r \in R \sm M$ so that $r + M \in R/M$ is a nonzero element. As $R/M$ is a field, there exists $x + M \in R/M$ such that $(r + M)(x + M) = 1 + M$. By the definitions of multiplication and addition in $R/M$, $1 - rx + M = M$ implying that $1 - rx \in M$.\\

	Let $r + M \in R/M$ be a nonzero element. Then $r \in R \sm M$ and so there exists $x \in \R$ such that $1 - rx \in M$. This means that $M = (1 - rx) + M = (1 + M) - (r + M)(x + M)$. Rearranging, we see that $(r + M)(x + M) = 1 + M$ meaning that $r + M$ has a multiplicative inverse in $R/M$. As $r + M$ was an arbitrary nonzero element, it follows that $R/M$ is a field and so $M$ must be a maximal ideal.
\end{proof}
\end{document}