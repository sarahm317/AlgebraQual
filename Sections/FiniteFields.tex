\documentclass[../AlgebraQualSolutions.tex]{subfiles}

\begin{document}

\subsection{Finite Fields}

\begin{prob}{(S20.RF2)}{S20.RF2}
Let $\F_3$ be the field with 3 elements.
\begin{enumerate}[(a)]
\item Prove that $K = \F_3[x]/(x^2 + 1)$ is a field.
\item How many elements does $K$ have?
\item Prove that $x+1$ generates the multiplicative group of non-zero elements in $K$.
\end{enumerate}
\end{prob}

\begin{proof}
	Let $f = x^2 + 1$. The only way for $f$ to be reducible over $\F_3$ is if $f$ has a root in $\F_3$. However this is impossible since $f(0) = 1$, $f(1) = 2$, and $f(2) = 2$. By Kronecker's Theorem, a polynomial ring quotiented out by an ideal generated by an irreducible polynomial is a field.
\end{proof}

\begin{solution}
	Let $\ol{1}$ and $\ol{x}$ denote the images of $1,x \in \F_3[x]$ under the quotient map. By Kronecker's Theorem, $\{\ol{1},\ol{x}\}$ form a basis for $\F_3[x]/(x^2 + 1)$ over $\F_3$. Therefore there are $3^2 = 9$ elements in $K$.
\end{solution}

\begin{proof}
	Since $|K| = 9$, the multiplicative group will have eight elements. Let $I = (x^2 + 1)$. Through algebraic manipulations, we see that 
		\[(x+1+I)^2 = 2x + I\]
		\[(x+1+I)^3 = 1 + 2x + I\]
		\[(x+1+I)^4 = 2 + I\]
		\[(x+1+I)^5 = 2 + 2x + I\]
		\[(x+1+I)^6 = x + I\]
		\[(x+1+I)^7 = 2 + x + I\]
		\[(x+1+I)^8 = 1 + I\]
	proving that $1 + x + I$ generates $K^*$.
\end{proof}

\begin{prob}{F14}{F14.G2}
	Prove that the multiplicative group of nohnzero elements of a finite field is cyclic.
\end{prob}

\begin{proof}
	Suppose that $F$ is a finite field and let $K = F^\times$ be the multiplicative group of nonzero elements in $F$. Since $F$ is finite, $K$ is finite -- let $|K| = n$. Let $a \in K$ be of maximal order, say $m$. By Lagrange's Theorem, $m | n$ and $m \leq n$. Suppose now that $b \in K$ is arbitrary and of order $r$.

	\begin{claim}
		There exists an element $c \in K$ of order $\lcm(r,m)$.

		\begin{proof}
			\textcolor{red}{Look for shorter proof of this!.}
		\end{proof}
	\end{claim}

	Then the order of $c$ is $\lcm(r,m) \geq m$. Because $m$ is the maximal order of elements in $K$, $\lcm(r,m) \leq m$ and therefore $\lcm{r,m} = m$. Since $r$ divides $\lcm{r,m} = m$ and $b$ is of order $r$, it follows that $b^m = 1$. As $b$ was an arbitrary element of $K$, every element of $K$ is a root of the polynomial $x^m - 1$. This polynomial has at most $m$ roots and therefore $n \leq m$. This implies that $n = m$ and whence $K$ is cyclic and generated by $a$.
\end{proof}

\end{document}