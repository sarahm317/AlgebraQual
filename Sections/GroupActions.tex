\documentclass[../AlgebraQualSolutions.tex]{subfiles}

\begin{document}

\subsection{Group Actions}

\begin{prob}{(S18.G3)}{S18.G3}
    \begin{enumerate}[(a)]
        \item Let $G$ be a finite group acting on a finite set $S$. Prove that the size of the orbit of a point is equal to the index  of its stablizer.
        \item Prove  the class equation:
            \[|G| = |Z(G)| + \sum_{g \in I}[G: C(g)]\]
        where $Z(G)$ is the center of $G$, $C(g)$ is the centralizer of $g$, and $I$ is a set with one representative from each nontrivial conjugacy class of $G$.
        \item Prove  that every finite $p$-group has nontrivial center.
    \end{enumerate}
\end{prob}

\begin{proof}
    Let $x^G = \{g\cdot x: g \in G\}$ denote the orbit of $x \in S$. Let $G_x = \{g \in G: g\cdot x = x\}$ denote the stabilizer subgroup of $x$. Let $A$ be the collection of left-cosets of $G_x$ in $G$. Define $f: x^G \to A$ by $f: g\cdot x \mapsto gG_x$.\\

    To see that $f$ is well-defined, suppose that $g\cdot x = h \cdot x$. Then, $(h\inv g)\cdot x = x$ implying that $h\inv g \in G_x$. Therefore, $h\inv g G_x = G_x$ and so $gG_x = hG_x$.\\

    Clearly $f$ is surjective: given any $gG_x \in A$, $g\cdot x \mapsto gG_x$. If $gG_x = hG_x$, then $h\inv g \in G_x$ and so $(h\inv g) \cdot x = x$. Therefore $h\cdot x = g\cdot x$ and so $f$ is injective.\\

    Since $f$ is a bijection from $A$ to $x^G$,
        \[|x^G| = [G: G_x].\]
\end{proof}

\begin{proof}
    Let $G$ act on itself by conjugation. That is, $g\cdot x = gxg\inv$. Then given $x \in G$,
        \[x^G = \{gxg\inv: g \in G\}\]
    and
        \[G_x = \{g \in G: gxg\inv = x\} = C(g).\]
    That is, the orbit of $x$ is the conjugacy class of $x$ in $G$ and the stablizer subgroup of $x$ is the centralizer of $x$.
    \begin{claim}
        The collection of conjugacy classes of $G$ partition $G$.

        \begin{proof}
            Notice that each $x \in G$ is in its own conjugacy class since $x = exe\inv$ where $e$ is the identity element. Suppose now that $y \in a^G \cap b^G$. Then
                \[gag\inv = y = hbh\inv\]
            for some $g,h \in G$. Rearranging this equation implies that $a \in b^G$. From here, it follows that $a^G \sq b^G$ and similarly $b^G \sq a^G$. That is, conjugacy classes are either disjoint or equal.
        \end{proof}
    \end{claim}

    Since the set of conjugacy classes of $G$ partition $G$,
        \[|G| = \sum_{x \in J} |x^G|\]
    where $J$ is a list consisting of one element from each conjugacy class of $G$. If a conjugacy class with representative $z$ is of size one, then
    $gzg\inv = z$ for each $g \in G$. That is, $z \in Z(G)$. If $z \in Z(G)$, then $\{gzg\inv: g \in G\} = \{z\}$. Therefore, the number of trivial conjugacy classes is equal to $|Z(G)|$. Therefore, if $I$ is the list of elements in $J$ that are not in $Z(G)$,
        \[|G| = |Z(G)| \sum_{x \in I} |x^G|.\]
    From the previous result,
        \[|G| = |Z(G)| + \sum_{x \in I} [G:G_x] = \sum_{x \in I}[G:C(x)].\]
\end{proof}

\begin{proof}
    Let $|G| = p^k$ for some prime $p$ and some $k \in \N$. The class equation implies that
        \[|G| = |Z(G)| + \sum_{x \in I}[G:C(x)]\]
    for some indexing set $I$ as defined previously. Each $[G:C(x)]$ is of size greater than one and must divide $|G|$. Therefore each $[G:C(x)]$ is some power of $p$. This means that $p$ must divide $|Z(G)|$ and so $|Z(G)|$ is nontrivial.
\end{proof}

\begin{prob}{F12}{F12.G2.a}
    Define the center of a group.
    \begin{enumerate}[(a)]
    \item Prove that if the order of $G$ is $p^k$ for some prime $p$ then $G$ has nontrivial center.
    \item Suppose that $p$ and $q$ are distinct primes. Prove that a non-abelian group of order $pq$ has trivial center.
    \end{enumerate}
    \end{prob}
    
    The center of a group $G$ is the subgroup
        \[Z(G) = \{g \in G: gx = xg \textrm{ for any $x \in G$} \} \]
    
    \begin{proof}
    Assume that $|G| = p^k$ for some prime $p$. If $k = 1$, then $G$ is cyclic and therefore abelian. In this case, $Z(G) = G$ and thus is nontrivial. Assume now that $k > 1$. If $G$ is abelian, $Z(G) = G$ and so $G$ has nontrivial center. If $G$ is non-abelian, $G - Z(G)$ is non-empty. Let $C_1, \ldots, C_n$ be the distinct conjugacy classes of elements in $G - Z(G)$. That is, for each $j \in \{1, \ldots, n\}$ let $x_j \in G - Z(G)$ such that $C_j = \{gx_jg\inv: g \in G\}$. Note that each $|C_j| > 1$ since a conjugacy class is a singleton if and only if the representative element is in $Z(G)$. The Orbit Stablizer Theorem implies that for each $j = 1, \ldots, n$,
        \[ |C_j| =  [G: C_G(x_j)] \]
    where $C_G(x_j) = \{g \in G: gx_jg\inv = x_j \}$ is the centralizer of $x_j$ in $G$. In particular, each $|C_j|$ divides $|G| = p^k$. Since $|C_j| > 1$, $p$ divides each $|C_j|$. The Class Equation states that 
        \[|Z(G)| = |G| - \sum_{j=1}^n [G: C_G(x_j)] = p^k -  \sum_{j=1}^n |C_j| .\]
    Because $p$ divides $p^k$ and $p$ divides each $|C_j|$, $p$ must also divide $|Z(G)|$. Thus $Z(G)$ is nontrivial.
    \end{proof}
    
    \begin{proof}
        See \ref{prob:F12.G2.b}.
    \end{proof}
    
    Note that \ref{prob:F18.G1} is Cauchy's Theorem and comes up for many Sylow-like problems. There's another version of the proof specifically for the case when $p = 2$ that does not involve group actions.
    
        \begin{prob}{F18.G1}{F18.G1}
        Let $G$ be a finite group with order that is divisible by a prime $p$. Prove that $G$ contains an element of order $p$.
        \end{prob}
        
        \begin{proof}
        Assume that $G$ is a finite group such that a prime $p$ divides $|G|$. Define a set $X \sq G^p$ as
            \[X = \{(x_1, \ldots, x_p) \in G^p: x_1 \cdots x_p = e\}. \]
        That is, $X$ is the set of all $p$-tuples of elements in $G$ where the product of the elements is the identity in $G$. Note that by choosing $x_1, \ldots, x_{p-1}$, $x_p$ is determined as
            \[x^p = (x_1 \cdots x_{p - 1})\inv. \]
        This means that $|X| = G^{p-1}$ and therefore $p$ must divide $|X|$.\\
        
        Next observe that if $x_1 \cdots x_p = e$, multiplying $x_1, \ldots, x_p$ in any order yields the identity. That is, if $(x_1, \ldots, x_p) \in X$, any permutation of this $p$-tuple is also in $X$. Therefore, we may let $\Z / p\Z$ act on $X$ via permutation. That is,
            \[1 \cdot (x_1, \ldots, x_p) \mapsto (x_p, x_1, \ldots, x_{p-1}). \]
        Since the order of $\Z / p\Z$ is prime, every stabilizer subgroup is either of size 1 or of size $p$. By the Order Stabilizer Theorem, this means that the order of an orbit is either 1 or $p$. Furthermore, the orbits of this action form a partition of $X$. Elements of $X$ that are in an orbit of size 1 must be of the form $(x, \ldots, x)$ where $x^p = e$. Since $(e, \ldots, e) \in X$ satisfies this condition, $e$ is in an orbit of size 1. Because orbits are of size 1 or $p$ and partition $X$, there exists some $x \neq e$ also in an orbit of size 1. If this were not the case, $p$ would not divide $X$, a contradiction. This chosen $x$ is of order $p$ since $x^p = e$.
        \end{proof}
        

\end{document}