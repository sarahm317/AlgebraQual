\documentclass[../AlgebraQualSolutions.tex]{subfiles}

\begin{document}

\subsection{General Group Theory}

\begin{prob}{(F03.G2)}{F03.G2}
	Suppose that $G$ is a finite group and let $K$ be a normal subgroup of $G$ such that $|K|$ and $[G:K]$ are coprime. Show that $K$ is the unique subgroup of order $|K|$.
\end{prob}

\begin{proof}
	Let $H \leq G$ be a subgroup such that $|H| = |K|$. Because $K$ is normal,  $HK$ is a subgroup of $G$. Lagrange's Theorem implies that $|HK|$ divides $|G|$. Notice that
		\[|HK| = \frac{|H| \cdot |K|}{|H \cap K|}\]
	and
		\[|G| = |K| \cdot [G:K].\]
	Therefore, $\frac{|H| \cdot |K|}{|H \cap K|}$ divides $|K| \cdot [G:K]$. By dividing each quantity by $|K|$, it follows that $\frac{|H|}{|H\cap K|}$ divides $[G:K]$. That is, there exists $n \in \N$ such that
		\[n \frac{|H|}{|H \cap K|} = [G:K].\]
	Multiplying both sides by $|H \cap K|$,
		\[n \cdot |H| = |H\cap K| \cdot [G:K]\]
	meaning that $|H|$ divides $|H\cap K| \cdot [G:K]$. But, $|H|$ and $[G:K]$ are coprime and therefore $|H|$ divides $|H \cap K|$. This implies that $H = H \cap K$ and so $H = K$.
\end{proof}

\begin{prob}{(S20.G3)}{S20.G3}
	\begin{enumerate}[(a)]
	\item Prove that the centralizer of an element is a subgroup.
	\item If $G$ is a finite group, prove that the number of elements in the conjugacy class divides the order of $G$.
	\end{enumerate}
	\end{prob}
	
	\begin{proof}
		Let $a \in G$ and let $C_G(a)$ denote the centralizer of $a$. If $e$ is the identity element in $G$, then $ea = ae$ and so $e \in C_G(a)$. Suppose now that $g,h \in C_G(a)$. Then, $ga = ag$ and $ha = ah$. Therefore, $(gh)a = g(ah) = a(gh)$ proving that $C_G(a)$ is closed under multiplication. Multiplying $ga = ag$ by $g\inv$ on the left and right of each side of the equation shows that $g\inv \in C_G(a)$ and so $C_G(a)$ is closed under inverses.
	\end{proof}
	
	\begin{proof}
		Let $a \in G$ and let $\rm{Cl}(a)$ denote the conjugacy class of $a$. Define $f: \{xC_G(a)\} \to \rm{Cl}(a)$ by $f: xC_G(a) \mapsto xax\inv$. To see that $f$ is well-defined, suppose that $xC_G(a) = yC_G(a)$. Then $x = yg$ for some $g \in C_G(a)$. This means that $g$ and $g\inv$ commute with $a$. Therefore,
			\[xax\inv = (yg)a(yg)\inv = ygag\inv y\inv = yay\inv\]
		and so $f$ is well-defined. Clearly $f$ is surjective. Now suppose that $xax\inv = yay\inv$. Then, $e = (x\inv y)a(x\inv y)\inv a\inv$ meaning that $x\inv y \in C_G(a)$. That is, $xC_G(a) = yC_G(a)$.\\
	
		Since $f$ is a bijection, the size of $\rm{Cl}(a)$ is equal to the number of left cosets of $C_G(a)$ in $G$ and therefore divides $|G|$, as desired.
	\end{proof}

\begin{prob}{(F17.G2)}{F17.G2}
	\begin{enumerate}[(i)]
		\item Let $G$ be a finite group and $H$ a subgroup of $G$. Prove that the order of $H$ divides the order of $G$.
		\item Let $\C^*$ be the multiplicative group of nonzero complex numbers. If $H$ is a subgroup of $\C^*$ of finite index, prove that $H = \C^*$.
	\end{enumerate}
\end{prob}

\begin{proof}
	This is an immediate consequence of Lagrange's Theorem. See \ref{prob:F12.G3}.
\end{proof}

\begin{proof}
	Let $H \leq \C^*$ and suppose that $[\C^*: H] = m$. Let $z \in \C$ be arbitrary. Then,
		\[z^mH = (zH)^m = H\]
	implying that $z^m \in H$. Now let $\omega \in \C^*$ be arbitrary and consider the polynomial $x^m - \omega$ over $\C$. This polynomial has some root, say $z \in \C^*$. This means that $z^m - \omega = 0$ or equivalently, $\omega = z^m$. But, $z^m \in H$ and so $\omega \in H$. Since $\omega$ was arbitrary, $\C^* = H$.
\end{proof}

\begin{prob}{(S16.G1)}{S16.G1}
	Explicitly describe two non-isomorphic, nonabelian groups of order 12.
\end{prob}

\begin{proof}
	Consider $A_4$ and $D_{12}$. Both groups are of order 12. However, $D_{12}$ contains an element of order 6 and $A_4$ does not. To see this, consider the elements in $A_4$:
		\begin{align*}
			&\perm{1}\\
			&\perm{1,2}\perm{3,4}\\
			&\perm{1,3}\perm{2,4}\\
			&\perm{1,4}\perm{2,3}\\
			&\perm{1,2,3}\\
			&\perm{1,3,2}\\
			&\perm{1,2,4}\\
			&\perm{1,4,2}\\
			&\perm{1,3,4}\\
			&\perm{1,4,3}\\
			&\perm{2,3,4}\\
			&\perm{2,4,3}
		\end{align*}
	Upon inspection, every element in $A_4$ is of order 1,2, or 3. In particular, there are no elements of order 6.
\end{proof}

\begin{prob}{(S12.G2)}{S12.G2}
	\begin{enumerate}[(a)]
		\item State (any version of) the Fundamental Theorem of finite abelian groups.
		\item Classify all abelian groups of order 144.
		\item Explain which group in part (b) is isomorphic to the group $\Z_4 \times \Z_{36}$.
	\end{enumerate}
\end{prob}

Suppose that $G$ is a finite abelian group. Then $G$ is isomorphic to a direct sum of the form
	\[\bigoplus_{i=1}^m \Z_{k_i}\]
where each $k_i$ is a power of some prime and $\prod_{i=1}^m k_i = |G|$. Note that the $k_i$ need not be powers of distinct primes.

\begin{solution}
Observe that $144 = 2^4 \cdot 3$. Therefore, the following are the abelian groups of order 144, up to isomorphism:

\begin{align*}
	\Z_{16} \times \Z_9 \\
	\Z_{16} \times \Z_3^2 \\
	\Z_8 \times \Z_2 \times \Z_9 \\
	\Z_8 \times \Z_2 \times \Z_3^2\\
	\Z_4 \times \Z_4 \times \Z_9 \\
	\Z_4 \times \Z_4 \times \Z_3^2\\
	\Z_4 \times \Z_2^2 \times \Z_9 \\
	\Z_4 \times \Z_2^2 \times \Z_3^2 \\
	\Z_2^4 \times \Z_9 \\
	\Z_2^4 \times \Z_3^2 \\
\end{align*}

The group $\Z_4 \times \Z_{36}$ is isomorphic to the group $\Z_4 \times Z_4 \times Z_9$. To see this, let $G = \Z_4 \times \Z_{36}$ and $H = \Z_4 \times \Z_{4} \times Z_9$ and consider the homomorphism $\varphi: H \to G$ given by
	\[\varphi(x,y,z) = (x,yz). \]
Clearly $\varphi$ is a homomorphism and is well-defined as $yz \in \{0,1, \ldots, 35\}$. The inverse of $\varphi$ is given by $(a,b) \mapsto(a,b\mod{4},b\mod{9})$ since 4 and 9 are relatively prime.
\end{solution}

\begin{prob}{S12.G1}{S12.G1}
	Prove that there are at least two non-isomorphic non-abelian groups of order 24.
\end{prob}

\begin{proof}
	Consider the groups $H = D_{24}$ and $G = \Z_2 \times D_{12}$. Since any dihedral group is nonabelian, both $H$ and $G$ must be nonabelian. However, $H$ contains an element of order 12. The order of an element $(x,y) \in G$ is the least common multiple of the order of $x \in \Z_2$ and $y \in D_{12}$. The highest order of an element in $\Z_2$ is 2 and the highest order of an element in $D_{12}$ is 6. Since the least common multiple of 2 and 6 is 6, it is impossible for $G$ to contain an element of order 12.
\end{proof}

	\begin{prob}{(F19.G1, F18.G2, F17.2, F12.G3)}{F12.G3}
	\begin{enumerate}[(a)]
	\item State and prove Lagrange's Theorem.
	\item Prove that a subgroup of a cyclic group is cyclic.
	\end{enumerate}
	\end{prob}
	
	Lagrange's Theorem States the following: If $G$ is a finite group and $H$ is a subgroup of $G$ then $[G:H]\cdot|H| = |G|$ where $[G:H]$ is the number of distinct left cosets of $H$ in $G$.

	\begin{proof}
		Suppose that $G$ is a finite group and $H \leq G$ is a subgroup.

		\begin{claim}
			The set of left cosets of $H$ in $G$ partitions $G$.

			\begin{proof}
				Note that any $g \in G$ is in the coset $gH$ since $g = g\cdot e$ where $e \in H$ is the identity element of the group. That is, each element of $G$ is in some left coset of $H$.\\

				Let $a,b \in G$ and suppose that $aH \cap bH \neq \varnothing$. Let $y \in aH \cap bH$ with $y = ah_1 = bh_2$ where $h_1,h_2 \in H$. Observe that $a = bh_1h_2\inv$ and $b = ah_2h_1\inv$. Then, for any $h \in H$,
					\[ah = bh_1h_2\inv h \in bH \]
				and
					\[bh = ah_2h_1\inv h \in aH.\]
				Therefore, $aH = bH$. Since any pair of left cosets are either disjoint or equal, it follows that the set of left cosets partitions $G$.
			\end{proof}
		\end{claim}

		\begin{claim}
			For any $g \in G$, $|H| = |gH|$.

			\begin{proof}
				Define $\varphi: H \to gH$ by $\varphi(h) = gh$. Clearly $\varphi$ is surjective. If $gh = gh'$, multiplying each side by $g\inv$ implies that $h = h'$ and so $\varphi$ is injective. As $\varphi$ is a bijection between $H$ and $gH$, $|H| = |gH|$ as desired. 
			\end{proof}
		\end{claim}

		As $G$ is finite, there are a finite number of left cosets of $H$ in $G$. Let $g_1H, \ldots, g_nH$ be the left cosets of $H$ in $G$. Each $g_iH$ has $|H|$ elements and each $g \in G$ is in some $g_iH$. Therefore,
			\[|G| = n|H| = [G:H]|H|,\]
		as desired.
	\end{proof}

	\begin{proof}
		Suppose that $G$ is a cylic group and $H \leq G$ a subgroup. Since $G$ is cyclic, there exists $a \in G$ such that $G = \langle a \rangle$. If $H$ is the trivial subgroup, then $H$ is generated by the identity. Otherwise, there exists $m \in \N$ such that $a^m \in H$ with $a^m \neq e$. Assume that $m$ is the least such positive integer. For any $b \in H$, there exists $n \in \N$ such that $b = a^n$ since $b \in G$. By the Division Algorithm, there exist integers $q,r \in \Z$ with $0 \leq r < m$ such that $n = mq + r$. Then,
			\[b = a^n = a^{mq+r} = a^{mq}a^r.\]
		Since $a^m \in H$, any power of $a^m$ is also contained in $H$. In particular, $a^{mq}, \left(a^{mq}\right)\inv \in H$. Therefore,
			\[a^r = b \left(a^{mq}\right)\inv \in H.\]
		Because $0 \leq r < m$ and $m$ is the minimal positive integer where $a^m \in H$, it follows that $r = 0$. Therefore, $n = mq$. This means that $b = a^n=a^{mq} = (a^m)^q$. Because $b$ was arbitrary, $H = \langle a^m \rangle$ proving that $H$ is cyclic.
	\end{proof}

	\begin{prob}{(F13.G1, S03.G1)}{F13.G1}
		A subgroup $H$ of a group $G$ is \emph{characteristic} if $\alpha(H) = H$ for any automorphism $\alpha$ of $H$.
		
		\begin{enumerate}[(a)]
			\item Prove that if $H$ is characteristic in $K$ and $K$ is characteristic in $G$, then $H$ is characteristic in $G$.
			\item Suppose now that $H$, $K$, and $G$ are groups with $H$ a normal subgroup of $K$ and $K$ a normal subgroup of $G$. Does this imply that $H$ is a normal subgroup of $G$?
		\end{enumerate}
	\end{prob}
	
	\begin{proof}
		Let $\varphi$ be an automorphism of $G$. Because $K$ is characteristic in $G$, $\varphi(K) = K$. Therefore, $\varphi$ restricts to an automorphism of $K$. Since $H$ is characteristic in $K$, $\varphi(H) = H$. But since $\varphi$ was an arbitrary automorphism of $G$, this means that $H$ is characteristic in $G$.
	\end{proof}
	
	\begin{solution}
	This need not be true. Let $G = S_4$, $H = \cyc{(12)(34),e}$, $K = \cyc{(12)(34), (13)(24),(14)(23), e}$. Upon inspection, $H$ is normal in $K$, $K$ is normal in $G$, but $H$ is not normal in $G$.
	\end{solution}

\end{document}