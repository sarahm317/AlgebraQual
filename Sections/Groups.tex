\documentclass[../AlgebraQualSolutions.tex]{subfiles}

\begin{document}

\section{General Group Theory}

	\begin{prob}{F12}{F12.G3}
	\begin{enumerate}[(a)]
	\item State and prove Lagrange's Theorem.
	\item Prove that a subgroup of a cyclic group is cyclic.
	\end{enumerate}
	\end{prob}
	
	Lagrange's Theorem States the following: If $G$ is a finite group and $H$ is a subgroup of $G$ then $[G:H]\cdot|H| = |G|$ where $[G:H]$ is the number of distinct left cosets of $H$ in $G$.

	\begin{proof}
		Suppose that $G$ is a finite group and $H \leq G$ is a subgroup.

		\begin{claim}
			The set of left cosets of $H$ in $G$ partitions $G$.

			\begin{proof}
				Note that any $g \in G$ is in the coset $gH$ since $g = g\cdot e$ where $e \in H$ is the identity element of the group. That is, each element of $G$ is in some left coset of $H$.\\

				Let $a,b \in G$ and suppose that $aH \cap bH \neq \varnothing$. Let $y \in aH \cap bH$ with $y = ah_1 = bh_2$ where $h_1,h_2 \in H$. Observe that $a = bh_1h_2\inv$ and $b = ah_2h_1\inv$. Then, for any $h \in H$,
					\[ah = bh_1h_2\inv h \in bH \]
				and
					\[bh = ah_2h_1\inv h \in aH.\]
				Therefore, $aH = bH$. Since any pair of left cosets are either disjoint or equal, it follows that the set of left cosets partitions $G$.
			\end{proof}
		\end{claim}

		\begin{claim}
			For any $g \in G$, $|H| = |gH|$.

			\begin{proof}
				Define $\varphi: H \to gH$ by $\varphi(h) = gh$. Clearly $\varphi$ is surjective. If $gh = gh'$, multiplying each side by $g\inv$ implies that $h = h'$ and so $\varphi$ is injective. As $\varphi$ is a bijection between $H$ and $gH$, $|H| = |gH|$ as desired. 
			\end{proof}
		\end{claim}

		As $G$ is finite, there are a finite number of left cosets of $H$ in $G$. Let $g_1H, \ldots, g_nH$ be the left cosets of $H$ in $G$. Each $g_iH$ has $|H|$ elements and each $g \in G$ is in some $g_iH$. Therefore,
			\[|G| = n|H| = [G:H]|H|,\]
		as desired.
	\end{proof}

	\begin{proof}
		Suppose that $G$ is a cylic group and $H \leq G$ a subgroup. Since $G$ is cyclic, there exists $a \in G$ such that $G = \langle a \rangle$. If $H$ is the trivial subgroup, then $H$ is generated by the identity. Otherwise, there exists $m \in \N$ such that $a^m \in H$ with $a^m \neq e$. Assume that $m$ is the least such positive integer. For any $b \in H$, there exists $n \in \N$ such that $b = a^n$ since $b \in G$. By the Division Algorithm, there exist integers $q,r \in \Z$ with $0 \leq r < m$ such that $n = mq + r$. Then,
			\[b = a^n = a^{mq+r} = a^{mq}a^r.\]
		Since $a^m \in H$, any power of $a^m$ is also contained in $H$. In particular, $a^{mq}, \left(a^{mq}\right)\inv \in H$. Therefore,
			\[a^r = b \left(a^{mq}\right)\inv \in H.\]
		Because $0 \leq r < m$ and $m$ is the minimal positive integer where $a^m \in H$, it follows that $r = 0$. Therefore, $n = mq$. This means that $b = a^n=a^{mq} = (a^m)^q$. Because $b$ was arbitrary, $H = \langle a^m \rangle$ proving that $H$ is cyclic.
	\end{proof}

	\begin{prob}{F13}{F13.G1}
		A subgroup $H$ of a group $G$ is \emph{characteristic} if $\alpha(H) = H$ for any automorphism $\alpha$ of $H$.
		
		\begin{enumerate}[(a)]
			\item Prove that if $H$ is characteristic in $K$ and $K$ is characteristic in $G$, then $H$ is characteristic in $G$.
			\item Suppose now that $H$, $K$, and $G$ are groups with $H$ a normal subgroup of $K$ and $K$ a normal subgroup of $G$. Does this imply that $H$ is a normal subgroup of $G$?
		\end{enumerate}
	\end{prob}
	
	\begin{proof}
		Let $\varphi$ be an automorphism of $G$. Because $K$ is characteristic in $G$, $\varphi(K) = K$. Therefore, $\varphi$ restricts to an automorphism of $K$. Since $H$ is characteristic in $K$, $\varphi(H) = H$. But since $\varphi$ was an arbitrary automorphism of $G$, this means that $H$ is characteristic in $G$.
	\end{proof}
	
	\begin{solution}
	This need not be true. Let $G = S_4$, $H = \cyc{(12)(34),e}$, $K = \cyc{(12)(34), (13)(24),(14)(23), e}$. Upon inspection, $H$ is normal in $K$, $K$ is normal in $G$, but $H$ is not normal in $G$.
	\end{solution}

\end{document}