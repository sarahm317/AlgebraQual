\documentclass[../AlgebraQualSolutions.tex]{subfiles}

\begin{document}

\subsection{Automorphisms}

\begin{prob}{F13.G2, S03.G2}{F13.G2}
	Prove that every finite group of order greater than two has a non-trivial automorphism.
\end{prob}

\begin{proof}
Suppose first that $G$ is a non-abelian group of order greater than two. Since $G$ is non-abelian, there exists an element $g \in G \sm Z(G)$. Then, the homomorphism $\varphi: G \to G$ given by $\varphi(x) = gxg\inv$ is non-trivial since there exists $x \in G$ where $gx \neq xg$. If $gxg\inv = gyg\inv$, then right and left multiplying by $g$ and $g\inv$, respectively, yields that $x = y$. Therefore $\varphi$ is a non-trivial automorphism.\\

Suppose now that $G$ is abelian. If $G$ is a 2-group, then $G \cong \Z_2^k$ for some $k \in \N$. Consider the automorphism $\varphi: \Z_2^k \to \Z_2^k$ given by 
	\[\varphi: (x_1,\ldots, x_k) \mapsto (x_2, x_1, \ldots, x_k).\]
Then $\varphi$ is a nontrivial automorphism, by construction. Composing $\varphi$ with the isomorphism between $G$ and $\Z_2^k$ yields the desired automorphism of $G$.\\

If $G$ is not a 2-group, then there exists some prime $p \neq 2$ that divides the order of $G$. By Cauchy's Theorem (see \ref{prob:F18.G1}), there exists an element $x \in G$ of order $p$. The map $\varphi: g \mapsto g\inv$ is then a non-trivial automorphism as $\varphi(x) \neq x$.
\end{proof}

\begin{prob}{F18.G3, F03.G3}{F18.G3}
    Let $\rm{Inn}(G)$ be the group of inner automorphisms of the group $G$ and $\rm{Aut}(G)$ the group of all automorphisms of $G$.
    \begin{enumerate}[(a)]
        \item Show that $\rm{Inn}(G)$ is normal in $\rm{Aut}(G)$.
        \item Show that if $Z(G)$ is the center of $G$, then $\rm{Inn}(G) \cong G/Z(G)$.
    \end{enumerate}
\end{prob}

\emph{The 2003 version of this question only asks part (a), but also requires that we prove that $\rm{Inn}(G)$ is a subgroup of $\rm{Aut}(G)$.}

\begin{proof}
    Let $\psi: G\to G$ be an automorphism. We must show that $\psi \rm{Inn}(G) \psi\inv = \rm{Inn}(G)$. Let $\varphi_g \in \rm{Inn}(G)$ be the inner automorphism such that $\varphi_g: x \mapsto gxg\inv$. For any $x \in X$,
            \[\psi\varphi_g\psi\inv(x) = \psi\left(g\psi\inv(x)g\inv\right) = \psi(g)x\psi(g)\inv\]
    where the last equality follows from the fact that $\psi$ is an automorphism. Therefore, $\psi\varphi_g\psi = \varphi_{\psi(g)} \in \rm{Inn}(G)$.
\end{proof}

\begin{proof}
    Define a map $F: G \to \rm{Inn}(G)$ by $F: g \mapsto \varphi_g$ where $\varphi_g: x \mapsto gxg\inv$. For any $g,h \in G$ and $x \in G$,

        \begin{align*}
            \varphi_g\circ\varphi_h(x) &= \varphi_g\left(hxh\inv\right)\\
            &= ghxh\inv g\inv\\
            &= (gh)x(gh)\inv\\
            &= \varphi_{gh}(x).
        \end{align*}
    Therefore, $F(gh) = \varphi_{gh} = \varphi_g\circ \varphi_h = F(g)F(h)$. By construction, $F$ is surjective. If $g \in Z(G)$, then $gxg\inv = xgg\inv = x$ for every $x \in G$. That is, $\varphi_g$ is the identity element in $\rm{Inn}(G)$ and so $Z(G) \sq \ker(F)$. If $\varphi_g$ is the identity map, then $gxg\inv = x$ for all $x \in G$. Equivalently, $g \in Z(G)$. Since $F(G) = \rm{Inn}(G)$ and $\ker(F) = Z(G)$, the First Isomorphism Theorem implies that $\rm{Inn}(G) \cong G/Z(G)$, as desired.
\end{proof}

\end{document}