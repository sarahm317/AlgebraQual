\documentclass[../AlgebraQualSolutions.tex]{subfiles}

\begin{document}

\subsection{General Rings and Fields}

\begin{prob}{(S03.RF1)}{S03.RF1}
    Let $V = \R^2$, regarded as a 2-dimensional vector space over $\R$. Let $L(V)$ denote the ring of all linear transformations from $V$ to $V$. Let $T \in L(V)$ be defined by $T(x,y) = (y,-x)$ and define
        \[A = \{S \in L(V): ST = TS\}.\]
    \begin{enumerate}[(a)]
        \item Prove that $A$ is a subring of $L(V)$.
        \item To which well-known ring is $A$ isomorphic? Give the isomorphism.
    \end{enumerate}
\end{prob}

\begin{proof}
    Let $I \in L(V)$ denote the identity transformation. Because $IT = TI$, it follows that $I \in A$. Suppose now that $R,S \in A$ and consider the linear transformation $R - S \in L(V)$. Then,
        \[(R-S)T = RT -ST = TR - TS = T(R - S)\]
    implying that $R - S \in A$. Similarly,
        \[(RS)T = R(ST) = R(TS) = (RT)S = (TR)S = T(RS)\]
    proving that $RS \in A$. As $A$ is nonempty, closed under addition, additive inverses, and multiplication, $A$ is a subring.
\end{proof}

\begin{claim}
    $A \cong \C$.
    
    \begin{proof}
        With respect to the standard basis for $\R^2$, the matrix of $T$ is
            \[M(T) = \begin{pmatrix} 0 & -1 \\ 1 & 0\end{pmatrix}.\]
        Given any $a,b \in \R$, define $S_{a,b} \in L(V)$ by $S_{a,b}(x,y) = (ax+by,-bx+ay)$. Note that $S_{a,b} \in A$ since
            \[ST(x,y) = S(y,-x) = (ay-bx, -by -ax) = T(ax+by, ay-bx) = TS(x,y)\]
        for each $(x,y) \in V$. Now define $\varphi: \C \to A$ by 
            \[\varphi(a+bi) = S_{a,b}.\]
        If $\varphi(a+bi) = \varphi(c+di)$, then $S_{a,b}(x,y) = S_{c,d}(x,y)$ for each $(x,y) \in V$. In particular,
            \[(a,-b) = S_{a,b}(1,0) = S_{c,d}(1,0) = (c,-d)\]
        implying that $a+bi=c+di$ and thus $\varphi$ is injective. To see that $\varphi$ is surjective, consider any $S \in A$. If the matrix of $S$ with respect to the standard basis is $M(S) = \begin{pmatrix} u & v \\ w & z\end{pmatrix}$, then
            \[\begin{pmatrix} 0 & -1 \\ 1 & 0\end{pmatrix}\begin{pmatrix} u & v \\ w & z\end{pmatrix} = \begin{pmatrix} u & v \\ w & z\end{pmatrix}\begin{pmatrix} 0 & -1 \\ 1 & 0\end{pmatrix}.\]
        Equivalently,
            \[\begin{pmatrix} v & -u \\ z & -w\end{pmatrix} = \begin{pmatrix} -w & -z \\ u & v\end{pmatrix}.\]
        Therefore,
            \[M(S) = \begin{pmatrix} u & v \\ w & z\end{pmatrix} = \begin{pmatrix} u & v \\ -v & u\end{pmatrix}\]
        meaning that $S = S_{u,v}$. That is, $\varphi(u +vi) = S_{u,v} = S$.\\

        The computations to verify that $\varphi$ is a homomorphism are straightforward and standard.
    \end{proof}
\end{claim}

\begin{prob}{(F17.RF3, S03.RF3)}{F17.RF3}
    Let $p$ be a prime and $\F_p$ the field of $p$ elements. Let $x$ be an indeterminate and let $R_1 = \F_p[x]/(x^2 - 2)$ and $\R_2 = \F_p[x]/(x^2 - 3)$. Determine whether or not the rings $R_1$ and $R_2$ are isomorphic when
    \begin{enumerate}[(a)]
        \item $p = 5$
        \item $p = 11$
    \end{enumerate}
\end{prob}


\begin{solution}
When $p = 5$, neither $x^2 - 2$ nor $x^2 - 3$ has any roots in $\F_5$. Therefore both of these polynomials are irreducible. By Kronecker's Theorem, $R_1$ and $R_2$ are both fields of order $5^2 = 25$. Since there is only one field of order 25, $R_1 \cong R_2$.\\

Now let $p = 11$. In this case, 5 is a root of $x^2 - 3$. Therefore, $R_2$ is not a field. However, $x^2 - 2$ is irreducible in $\F_{11}$ since it has no roots. Therefore, $R_1$ is a field and $R_1 \not\cong R_2$.
\end{solution}

\begin{prob}{(S20.RF2), (F14.G2)}{S20.RF2}
Let $K$ be a field. Show that every finite subgroup of $K^\times$ is cyclic.
\end{prob}

\begin{proof}
	Let $G \leq K^\times$ be a finite subgroup. Then $G$ is a finite abelian group and can therefore can be written as the direct product of its Sylow subgroups, each of some prime power order. By the Chinese Remainder Theorem, this direct product is cyclic if and only if each of the Sylow subgroups is cyclic. Therefore, without loss of generality, assume that $|G| = p^k$.\\

	Let $m$ denote the maximal order of the elements in $G$. That is, $m \leq p^k$ and $g^m - 1 = 0$ for each $g \in G$. This  means that each $g \in G$ is a root of $x^m - 1$. Since $G$ is a field, this polynomial has at most $m$ roots. Therefore $m = p^k$ and so $G$ has an element of order $m$ which generates the subgroup.
\end{proof}

\begin{prob}{F16.RF3}{F16.RF3}
    Let $p$ be a prime and $R_p$ the set of all $2 \times 2$ matrices of the form $\begin{pmatrix} a & b\\ -b & a\end{pmatrix}$ where $a,b \in \Z_p$.

    \begin{enumerate}[(i)]
        \item Show that $R_p$ is a commutative ring with identity.
        \item Show that $R_7$ is a field.
        \item Show that $R_{13}$ is not a field.
    \end{enumerate}
\end{prob}

\begin{proof}
    Let $S,T \in R_p$ be arbitrary. Both $ST$ and $S + T$ are in $R_p$, by properties of modular arithmetic. Notice that $ST = TS$, following from the definition of matrix multiplication and the commutativity of $\Z_p$. Similarly, $S + T = T + S$. The additive inverse of $S$ is $-S$ which is in $R_p$. The multiplicative identity in $R_p$ is the $2 \times 2$ identity matrix and the additive identity is the $2 \times 2$ zero matrix.
\end{proof}

\begin{proof}
    From (i), $R_7$ is a commutative ring with identity. It remains to show that each element of $R_7$ has a multiplicative inverse. Let $\begin{pmatrix} a & b\\ -b & a\end{pmatrix} \in R_7$ be nonzero. Then at most one of $a$ and $b$ can be zero. Consider the quantity $a^2 + b^2 \in R_7$. In $R_7$, $a^2,b^2 \in \{0, 1, 2, 4\}$. At most one of $a^2$ and $b^2$ can be zero and upon inspection this implies $a^2 + b^2 \neq 0$. Therefore, $a^2 + b^2$ has an inverse in $R_7$. The matrix 
        \[\begin{pmatrix}
            a \left(a^2 + b^2\right)\inv & -b \left(a^2 + b^2\right)\inv\\
            b\left(a^2 + b^2\right)\inv & a\left(a^2 + b^2\right)\inv
        \end{pmatrix}\]
    is therefore well-defined and is the multiplicative inverse of the original matrix.
\end{proof}

\begin{solution}
    Consider the matrix $\begin{pmatrix} 2 & 3\\ -3 & 2\end{pmatrix} \in R_{13}$. Because $2^2 + 3^2 = 0$ in $\R_{13}$, there is no multiplicative inverse for the matrix.
\end{solution}

\end{document}