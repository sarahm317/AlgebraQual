\documentclass[../AlgebraQualSolutions.tex]{subfiles}

\begin{document}

\subsection{General Rings and Fields}

\begin{prob}{(S20.RF2), (F14.G2)}{S20.RF2}
Let $K$ be a field. Show that every finite subgroup of $K^\times$ is cyclic.
\end{prob}

\begin{proof}
	Let $G \leq K^\times$ be a finite subgroup. Then $G$ is a finite abelian group and can therefore can be written as the direct product of its Sylow subgroups, each of some prime power order. By the Chinese Remainder Theorem, this direct product is cyclic if and only if each of the Sylow subgroups is cyclic. Therefore, without loss of generality, assume that $|G| = p^k$.\\

	Let $m$ denote the maximal order of the elements in $G$. That is, $m \leq p^k$ and $g^m - 1 = 0$ for each $g \in G$. This  means that each $g \in G$ is a root of $x^m - 1$. Since $G$ is a field, this polynomial has at most $m$ roots. Therefore $m = p^k$ and so $G$ has an element of order $m$ which generates the subgroup.
\end{proof}

\begin{prob}{F16.RF3}{F16.RF3}
    Let $p$ be a prime and $R_p$ the set of all $2 \times 2$ matrices of the form $\begin{pmatrix} a & b\\ -b & a\end{pmatrix}$ where $a,b \in \Z_p$.

    \begin{enumerate}[(i)]
        \item Show that $R_p$ is a commutative ring with identity.
        \item Show that $R_7$ is a field.
        \item Show that $R_{13}$ is not a field.
    \end{enumerate}
\end{prob}

\begin{proof}
    Let $S,T \in R_p$ be arbitrary. Both $ST$ and $S + T$ are in $R_p$, by properties of modular arithmetic. Notice that $ST = TS$, following from the definition of matrix multiplication and the commutativity of $\Z_p$. Similarly, $S + T = T + S$. The additive inverse of $S$ is $-S$ which is in $R_p$. The multiplicative identity in $R_p$ is the $2 \times 2$ identity matrix and the additive identity is the $2 \times 2$ zero matrix.
\end{proof}

\begin{proof}
    From (i), $R_7$ is a commutative ring with identity. It remains to show that each element of $R_7$ has a multiplicative inverse. Let $\begin{pmatrix} a & b\\ -b & a\end{pmatrix} \in R_7$ be nonzero. Then at most one of $a$ and $b$ can be zero. Consider the quantity $a^2 + b^2 \in R_7$. In $R_7$, $a^2,b^2 \in \{0, 1, 2, 4\}$. At most one of $a^2$ and $b^2$ can be zero and upon inspection this implies $a^2 + b^2 \neq 0$. Therefore, $a^2 + b^2$ has an inverse in $R_7$. The matrix 
        \[\begin{pmatrix}
            a \left(a^2 + b^2\right)\inv & -b \left(a^2 + b^2\right)\inv\\
            b\left(a^2 + b^2\right)\inv & a\left(a^2 + b^2\right)\inv
        \end{pmatrix}\]
    is therefore well-defined and is the multiplicative inverse of the original matrix.
\end{proof}

\begin{solution}
    Consider the matrix $\begin{pmatrix} 2 & 3\\ -3 & 2\end{pmatrix} \in R_{13}$. Because $2^2 + 3^2 = 0$ in $\R_{13}$, there is no multiplicative inverse for the matrix.
\end{solution}

\end{document}