\documentclass[../AlgebraQualSolutions.tex]{subfiles}

\begin{document}

\subsection{Finite Fields}

\begin{prob}{(F03.RF1)}{F03.RF1}
Give an explicit construction for a field $F$ of 9 elements.
\end{prob}
\begin{proof}
	Let $K = \mathbb{F}_3 \cong \Z/3\Z$ be the field of 3 elements. Consider the polynomial $f(x) = x^2 + 1 \in K[x]$. Since $f$ is a degree-two polynomial, $f$ is reducible if and only if $f$ has a root in $K$. But, $f(0) = 1, f(1) = 2, f(-1) = 2$. Thus, $f$ is irreducible in $K[x]$ and by Kronecker's Theorem, $F = K[x]/(x^2 + 1)$ is a field of $3^2 = 9$ elements. The elements of $F$ are
		\[0 + (x^2 + 1) \quad 1 + (x^2 + 1) \quad 2 + (x^2 + 1)\]
		\[x + (x^2 + 1) \quad x + 1 + (x^2 + 1) \quad x + 2 + (x^2 + 1)\]
		\[2x + (x^2 + 1) \quad 2x + 1 + (x^2 + 1) \quad 2x + 2 + (x^2 + 1)\]
\end{proof}

\begin{prob}{(S04.RF1)}{S04.RF1.c}
	Let $F$ be a finite field. Using only vector space theory, prove that if $|F| = p^m$ and $K$ is an extension field  of $F$ with $|K|=p^n$, then $m$ divides $n$.
\end{prob}

\begin{proof}
	Note first that both $F$ and $K$ contain a copy of $\F_p$.
		\begin{claim}
			Let $F$ be a finite field of characteristic $p$. Then, $|F|=p^n$ if and only if $[F:\F_p] = n$.

			\begin{proof}
				Since $\rm{char}(F) = p$, there exists a subfield $L$ of $F$ that is isomorphic to $\F_p$. Therefore, $F$ can be viewed as an $\F_p$-vector space. Let $\{x_1,\ldots, x_n\}$ be an $\F_p$-basis for $F$. A basic counting argument and the definition of the degree of an extension then implies that $[F:\F_p] = n$ if and only if $|F|=p^n$.
			\end{proof}
		\end{claim}

	From the claim, $[F:\F_p] = m$ and $[K:\F_p] = n$. As $F$ is a subfield of $K$, there exists an $F$-basis, say $\{k_1, \ldots, k_r\}$ for $K$. That is, for each $k \in K$ there exist unique $b_1,\ldots b_r \in F$ such that
		\[k = b_1k_1 + \cdots b_rk_r.\]
	Similarly, there exists an $\F_p$-basis, say $\{f_1,\ldots, f_m\}$, for $F$. Then each $b_i$ can be written uniquely in the form 	
		\[b_i = a_1f_1 + \cdots + a_mf_m\]
	with each $a_1,\ldots,a_m \in \F_p$. A counting argument can then be used to see that $p^n = |K| = \left(p^m\right)^r$. That is, $p^n = p^mr$ and therefore $m$ divides $n$, as desired.
\end{proof}


\begin{prob}{(S04.RF1.ab, F18.RF1)}{F18.RF1}
	\begin{enumerate}[(a)]
		\item Define the characteristic of a field. Prove that a finite field must have prime characteristic $p$ and deduce that it has $p^k$ elements for some $k \in \N$.
		\item Prove  that a finite field cannot be algebraically closed.
		\item Construct a field of 125 elements.
	\end{enumerate}
\end{prob}

\begin{proof}
	Let $F$ be a finite field. Since $F$ is finite, the cyclic additive  group generated by 1 is finite. Therefore, the characteristic of $F$ is finite.\\
	
	Seeking a contradiction, suppose that the characteristic of $F$ is some composite number $m\cdot n$. Then,
		\[0 = (mn)\cdot 1 = m\cdot (1 + \cdots + 1) = (1 + \cdots + 1)_{\textrm{$m$ times}} (1 + \cdots + 1)_{\textrm{$n$ times}}.\]
	But this is a contradiction since $F$ cannot have any zero divisors and $mn$ is defined to be minimal. Therefore the characteristic of $F$ is prime.\\

	Let $p$ be the prime characteristic of $F$. Then the prime field of $F$ is isomorphic to $\F_p$. This means that $[F: \F_p] = k$ for some $k \in \N$. Viewing $F$ as a vector space over $\F_p$, it follows that $F$ has $p^k$ elements.
\end{proof}

\begin{proof}
	Let $F$ be a  finite field of order $p^k$. Enumerate the elements of $f$ as $\alpha_1,\ldots, \alpha_{p^k}$. Consider the polynomial in $F[x]$ given by 
		\[f(x) = (x-\alpha_1)\cdots (x-\alpha_{p^k}) + 1.\]
	Then $f(\alpha_j) = 1$ for each $j = 1,\ldots,p^k$. Since $f$ is a polynomial over $F$ with no roots in $F$, $F$ is not algebraically closed.
\end{proof}

\begin{solution}
	Let $\F_5$ be the finite field with 5 elements and consider the polynomial $f(x) = x^3 + x^2 + x + 3$. If $f$ is reducible, then $f$ must have a root in $\F_5$. However, $f(0) = 3, f(1) = 1, f(2) = 2, f(3) = 2$, and $f(4) = 2$. Therefore $f$ is irreducible. By Kronecker's Theorem,
		\[F = \F_5/(f)\]
	is a field and has $5^3$ elements.
\end{solution}

\begin{prob}{(S20.RF2)}{S20.RF2}
Let $\F_3$ be the field with 3 elements.
\begin{enumerate}[(a)]
\item Prove that $K = \F_3[x]/(x^2 + 1)$ is a field.
\item How many elements does $K$ have?
\item Prove that $x+1$ generates the multiplicative group of non-zero elements in $K$.
\end{enumerate}
\end{prob}

\begin{proof}
	Let $f = x^2 + 1$. The only way for $f$ to be reducible over $\F_3$ is if $f$ has a root in $\F_3$. However this is impossible since $f(0) = 1$, $f(1) = 2$, and $f(2) = 2$. By Kronecker's Theorem, a polynomial ring quotiented out by an ideal generated by an irreducible polynomial is a field.
\end{proof}

\begin{solution}
	Let $\ol{1}$ and $\ol{x}$ denote the images of $1,x \in \F_3[x]$ under the quotient map. By Kronecker's Theorem, $\{\ol{1},\ol{x}\}$ form a basis for $\F_3[x]/(x^2 + 1)$ over $\F_3$. Therefore there are $3^2 = 9$ elements in $K$.
\end{solution}

\begin{proof}
	Since $|K| = 9$, the multiplicative group will have eight elements. Let $I = (x^2 + 1)$. Through algebraic manipulations, we see that 
		\[(x+1+I)^2 = 2x + I\]
		\[(x+1+I)^3 = 1 + 2x + I\]
		\[(x+1+I)^4 = 2 + I\]
		\[(x+1+I)^5 = 2 + 2x + I\]
		\[(x+1+I)^6 = x + I\]
		\[(x+1+I)^7 = 2 + x + I\]
		\[(x+1+I)^8 = 1 + I\]
	proving that $1 + x + I$ generates $K^*$.
\end{proof}

\end{document}