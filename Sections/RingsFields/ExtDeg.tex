\documentclass[../../AlgebraQualSolutions.tex]{subfiles}

\begin{document}

\subsection{Field Extensions}

\begin{prob}{(F04.RF1)}{F04.RF1}
	Suppose that $K = F(u)$ is a finite field extension of $F$ such that the degree $[K:F]$ is odd. Show that $K = F(u^2 - u)$.
\end{prob}

\begin{proof}
	Since $F(u)$ is a field containing $u$, $u^2 - u \in F(u)$. Thus, $F(u^2 - u) \sq F(u)$ and we may consider the following diagram:

	\[\begin{tikzcd}
		K = F(u) \ar[dd, no head,"2n - 1"'] \ar[dr, no head] & \\
		 & F(u^2 - u) \ar[dl, no head] \\
		 F &
	\end{tikzcd}\]
	for some $n \in \N$. Seeking a contradiction, suppose that $[F(u): F(u^2 - u)] > 1$. In particular, this means that $u \not\in F(u^2 - u)$. The minimal polynomial for $u$ over $F(u^2 - u)$ is 
		\[f(x) = x^2 - x - (u^2 - u)\]
	since the degree of the minimal polynomial is at least degree two and $f$ has $u$ as a root. Therefore, $[F(u): F(u^2 - u)] = 2$. By the Tower Rule, this means that 2 must divide $[K: F]$, contradicting the assumption that the degree was odd. Therefore, $[F(u): F(u^2 - u)] = 1$, or equivalently, $F(u) =  F(u^2 - u)$ as desired.
\end{proof}


\begin{prob}{(F03.RF2)}{F03.RF2}
	Suppose that $\alpha \neq 0$ is algebraic over $\Q$ and $[\Q(\alpha): \Q]$ is odd. Show that $\Q(\alpha) = \Q(\alpha + \alpha\inv)$. Must this be true without the assumption that $[\Q(\alpha): \Q]$ is odd?
\end{prob}

\begin{proof}
	Since $\Q(\alpha)$ is a field containing $\alpha$, it also must contain $\alpha + \alpha\inv$ and so $\Q(\alpha + \alpha\inv) \sq \Q(\alpha)$. Tower Rule implies that
		\[[\Q(\alpha): \Q(\alpha + \alpha\inv):] \cdot [\Q(\alpha + \alpha\inv): \Q] = [\Q(\alpha): \Q].\]
	Notice that $\alpha$ is a root of the polynomial $x^2 - (\alpha + \alpha\inv)x + 1 \in \Q(\alpha + \alpha\inv)[x]$. Therefore, $[\Q(\alpha + \alpha\inv): \Q]$ is either 1 or 2. But $[\Q(\alpha): \Q]$ is odd and therefore $[\Q(\alpha + \alpha\inv): \Q] = 1$. This implies that $\Q(\alpha) = \Q(\alpha + \alpha\inv)$, as desired.
\end{proof}

If the assumption that $[\Q(\alpha): \Q]$ is odd is removed, this statement is false. Consider the extension $\Q(i)$ over $\Q$. This is a degree 2 extension, but $\Q(i + i\inv) = \Q(i + (-i)) = \Q \neq \Q(i)$.

\begin{prob}{(F17.RF2, S03.RF2)}{F17.RF2}
	\begin{enumerate}[(i)]
		\item State Eisenstein's irreducibility criterion for polynomials.
		\item Find $[\Q(\sqrt 2): \Q]$ and $[\Q(3^{1/3}): \Q]$.
		\item Using (ii), prove that $\alpha = \sqrt{2}+3^{1/3}$ is irrational.
	\end{enumerate}
\end{prob}

\begin{proof}
	The minimal polynomial for $\sqrt2$ over $\Q$ is $x^2 - 2$ and therefore $[\Q(\sqrt 2): \Q] =2$. Similarly, the minimal polynomial for $3^{1/3}$ over $\Q$ is $x^3 - 3$ and so $[\Q(3^{1/3}): \Q] = 3$.
\end{proof}

\begin{proof}
	Seeking a contradiction, suppose that $\alpha \in \Q$.

	\[
		\begin{tikzcd}[column sep=1cm]
		  & \Q(\sqrt2,3^{1/3}) & \\
		  \Q(\sqrt2) \ar[ru, no head,"m"]& & \ar[lu, no head,"n"'] \Q(3^{1/3}) \\
		  & \Q \ar[lu, no head, "2"] \ar[ru, no head, "3"']&
		\end{tikzcd}
		\]
	Because $\alpha \in \Q$, $3^{1/3} = \alpha - \sqrt2 \in \Q(\sqrt2)$. Therefore, $[\Q(\sqrt2 + 3^{1/3}): \Q(\sqrt2)] = m = 1$. Similarly, $\sqrt{2} = \alpha - 3^{1/3} \in \Q(3^{1/3})$ and so $[\Q(\sqrt2 + 3^{1/3}): \Q(3^{1/3})] = n = 1$. However, the Tower Rule implies that $2 = m\cdot2 = n\cdot 3 = 3$, a contradiction. 
\end{proof}


\end{document}