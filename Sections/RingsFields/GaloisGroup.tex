\documentclass[../../AlgebraQualSolutions.tex]{subfiles}

\begin{document}

\subsection{Galois Groups}

\begin{prob}{(F14.RF2)}{F14.RF2}
	\begin{enumerate}[(a)]
		\item Find a monic polynomial $f(x) \in \Q[x]$ that has $\sqrt{1 + \sqrt2}$ as a root.
		\item Find the splitting field $K$ of $f(x)$ over $\Q$.
		\item Find the Galois group of $K$ over $\Q$.
	\end{enumerate}
\end{prob}

\begin{solution}
	Consider the polynomial $x^4 - 2x^2 - 1$. Upon inspection, this polynomial has $\sqrt{1 + \sqrt2}$ as a root and is monic.
\end{solution}

\begin{solution}
	Let $f(x) = x^4 - 2x^2 - 1$. The roots of $f$ are
		\[\sqrt{1 + \sqrt2}\]
		\[\sqrt{1 - \sqrt2}\]
		\[-\sqrt{1 + \sqrt2}\]
		\[-\sqrt{1 - \sqrt2}\]
	Notice that since $1 - \sqrt2<0$, the splitting field for $f$ must contain complex values. Let $a = \sqrt{1 + \sqrt2}$ and $b = \sqrt{1 - \sqrt2}$. Define $c = \sqrt{\sqrt2 - 1}$ so that $b = ic$. Upon inspection, $a\inv = c$. Define $K = \Q(a, i)$. Then $K$ contains $\pm a$, by additive closure. Since $a\inv = c$, $i a\inv = ic = b$ must also be in $K$. Therefore $K$ contains all four roots of $f$ and thus is a splitting field.
\end{solution}

\begin{solution}
	Consider the following diagram:

	\[
		\begin{tikzcd}
			& \Q(a,i) = K & \\
			\Q(a) \ar[ur,no head, "2"] & & \Q(i) \ar[ul,no head, "4"']\\
			& \Q \ar[ul, no head, "4"] \ar[ur, no head, "2"'] \ar[uu, dashed, no head, "8"] &
		\end{tikzcd}
		\]

	From (a), we know that $a$ is a root of an irreducible degree 4 polynomial and so $[\Q(a): \Q] = 4$. Since both $\Q$ and $\Q(a)$ are real-valued fields, $[\Q(a,i): \Q(a)] = 2$ and $[Q(i): \Q] = 2$. The remaining degrees in the diagram follow directly from Tower Rule. \\
\end{solution}
\begin{solution}
	From (b), it follows that $|G(K:\Q)| = 8$ since the order of the Galois group equals the degree of the splitting field over the base field.\\
	
	Let $f = x^4 - 2x^2 - 1$. Then $f$ is irreducible over $\Q$ and has $a$ as a root. For simplicity, let $a_1 = a,a_2 = -a ,a_3 = b,a_4 = -b$ denote the roots of $f$. By the Extension Lemma the identity map on $\Q$ can be extended to $\varphi_j: \Q(a) \to \Q(a_j)$ where $\varphi_j: a \mapsto a_j$. Apply the Extension Lemma again to each $\varphi_j$ to obtain $\varphi_{j1}: \Q(a,i) \to \Q(a_j,i)$ where $\varphi_{j1}: i\mapsto i$ and $\varphi_{j2}: \Q(a,i) \to \Q(a_j,i)$ where $\varphi_{j2}: i \mapsto -i$. Here, $\pm i$ are the roots of the irreducible polynomial $x^2 + 1$ over $\Q(a)$. Upon inspection, $K = \Q(a,i) = \Q(a_j,i)$ and therefore each of the eight $\varphi{jk}$ is a $\Q$-automorphism. Since $|G(K:\Q)| = 8$, these are all of the automorphisms.\\

	The automorphism that sends $a$ to $b$ and $i$ to $i$ is of order four. Furthermore, the Galois group is nonabelian since the automorphism of order four does not commute with the automorphism that sends $a$ to itself and $i$ to $-i$. Therefore $G(K:\Q) \cong D_8$.
\end{solution}

\begin{prob}{(S11.RF3), (S18.RF3)}{S11.RF3}
    Let $F$ be a splitting field of the polynomial $f(x) = x^4 - 2 \in \Q[x]$ over $\Q$. Find the degree $[F: \Q]$ and determine the Galois group of the extension $\Q \sq F$ up to isomorphism.
\end{prob}

\begin{proof}
    Let $a = \sqrt[4]{2}$ and  observe that $\{a,ai,-a,-ai\}$ are all roots of $f$. Since $\deg(f) = 4$, these are the only roots of $f$. By Eisenstein's Criterion, $f$ is irreducible over $\Q$. Therefore $f$ is the minimal polynomial for $a$ over $\Q$ and so $[\Q(a),\Q] = 4$. Since $\Q(a)$ is a real-valued field, $i \not\in \Q(a)$ and so the degree of the minimal polynomial for $i$ over $\Q(a)$ is at least 2. But, $i$ is a root of $x^2 + 1 \in \Q(a)[x]$ and so $[\Q(a,i): \Q(a)] = 2$. By the Tower Rule, $[\Q(a,i): \Q] = 4$. Since $\Q(a,i)$ contains all the roots of $f$, it is a splitting field for $f$ over $\Q$ and thus is isomorphic to $F$. So, $[F: \Q] = 4$ as well.\\

    Since $f(a) = 0$ and $f$ is irreducible over $\Q$ The Extension Lemma yields isomorphisms $\varphi_k: \Q(a) \to \Q(ai^k)$ where $\varphi:a \mapsto ai^k$ for $k = 0,1,2,3$. Next define $g(x) = x^2 + 1 \in \Q(a)$ and note that $g(i) = 0$ and $g$ is irreducible over $\Q(a)$ as the roots are both complex. Again, by the Extension Lemma each $\varphi_k$ may be extended to $\varphi_{k1}: \Q(a,i) \to \Q(ai^k, i)$ or $\varphi_{k2}: \Q(a,i) \to \Q(ai^k,-i)$ where
        \[\varphi_{k1}: \begin{cases}
            a \mapsto ai^k\\
            i \mapsto i
        \end{cases}\]
    and
        \[\varphi_{k2}: \begin{cases}
        a \mapsto ai^k\\
        i \mapsto -i
    \end{cases}\]
    Observe that this accounts for 8 $\Q$-automorphisms of $\Q(a,i)$. Since $f$ has 4 roots and $g$ has 2 roots and $\Q$-automorphisms must permute roots of polynomials over $\Q$, there are at most 8 total $\Q$-automorphisms. Thus $|G(\Q(a,i): \Q)| = 8$. Upon inspection, $G = G(\Q(a,i): \Q)$ is nonabelian and contains an element of order 4, $\varphi_{11}$. That is, $G \cong D_8$. 
\end{proof}

\end{document}