\documentclass[../../AlgebraQualSolutions.tex]{subfiles}

\begin{document}

\subsection{Polynomial Rings}

\begin{prob}{(S20.RF4)}{S20.RF4}
    Prove that the polynomial $x^3 - 2$ is irreducible over the field $\Q(i)$.
\end{prob}

\begin{proof}
    Let $f = x^3 - 2$. Notice that $f$ is irreducible over $\Q$ by Eisenstein's Theorem. If $\zeta$ denotes a primitive 3rd root of unity, then the three roots of $f$ are $a,a\zeta, a\zeta^2$. That is, $[\Q(a): \Q] = [\Q(a\zeta): \Q] = [\Q(a\zeta^2): \Q] = 3$\\
    
    If $f$ were to be reducible over $\Q(i)$, then $f$ would have a root in $\Q(i)$. Seeking a contradiction, let $\alpha \in \Q(i)$ be a root of $f$. Then, $\Q(\alpha) \sq \Q(i)$ is a subfield. Since $x^2 + 1$ is irreducible over $\Q$ and has $i$ as a root, $[\Q(i): \Q] = 2$. By Tower Rule, $2 = [\Q(i): \Q] = [\Q(i): \Q(\alpha)]\cdot[\Q(\alpha): \Q]$. But, $a$ is a root of $f$ and therefore $[\Q(\alpha): \Q]$, a contradiction.
\end{proof}

\begin{prob}{(F20.RF1)}{F20.RF1}
    The polynomial $x^3 - x$ has six roots in the ring $\Z/6\Z$. Find a sufficient condition on a commutative ring $R$ which ensures that the number of roots of a polynomial with coefficients in $R$ cannot exceed its degree and justify your assertion.
    \end{prob}
    
    \begin{claim}
        If $R$ is a field and $f \in R[x]$ is a nonzero polynomial of degree $n$, then $f$ has at most $n$ roots in $R$.
    
        \begin{proof}
            We proceed by induction on the degree of $f$. If $\deg(f) = 0$, then $f$ is constant and nonzero, meaning it has no zeros. Assume that $\deg(f) = 1$ so that $f(x) = ax + b$ for some $a,b \in R$. If $r,s$ are both roots of $f$, then
                \[ar + b = as + b\]
            and therefore $r = s$. That is, $f$ has at most one root in $R$.\\
            
            Now let $\deg(f) = n$ and assume that the result holds for any polynomial $p$ of degree less than $n$. If $f$ has no zeros in $R$, then the result holds. Otherwise, suppose that $r$ is a root of $f$ in $R$. Then, $f(x) = (x - r)q(x)$ where $q \in R[x]$ is a polynomial of degree $n - 1$. By the inductive hypothesis, $q$ has at most $n - 1$ roots and therefore $f$ has at most $n$ roots.
        \end{proof}
    \end{claim}

\end{document}