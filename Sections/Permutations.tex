\documentclass[../AlgebraQualSolutions.tex]{subfiles}

\begin{document}

\section{Permutation Groups}

\begin{prob}{S18.G1}{S18.G1}
    Show that if $G$ is a group of order $2k$ where $k$ is odd, then $G$ has a subgroup of index 2.
\end{prob}

\begin{proof}
    Enumerate the elements of $G$ with values $\{1,\ldots, 2k\}$. For each $a \in G$, let $\varphi_a: G \to G$ be left multiplication by $a$. Identify $\varphi_a$ with the permutation in $S_{2k}$ corresponding to the image of $\varphi_a$. Then define $\varphi: G \to S_{2k}$ by $\varphi: a \mapsto \varphi_{a} \in S_{2k}$. By construction, $\varphi$ is injective and a homomorphism meaning that $G$ is isomorphic to some subgroup $H$ of $S_{2k}$.\\

    By Cauchy's Theorem (or \ref{prob:F12.G1}), $G$ has an element $x$ of order 2 since the order of $G$ is divisible by two. Consider $\varphi(x) \in H$. Since $x$ is of order 2 and $\varphi$ is an isomorphism, $\varphi(x)$ must also be of order 2. That is, $\varphi(x)$ is the product of some number of disjoint transpositions. Since $|G| = 2k$, there are at most $k$ disjoint transpositions in $\varphi(x)$. If $\varphi(x)$ is the product of less than $k$ transpositions, then there exists $g \in G$ such that $xg = g$. However, this would imply that $x = e$ which contradicts the fact that $x$ is of order 2. Therefore $\varphi(x)$ is the product of exactly $k$ disjoint transpositions. Since $k$ is odd, $H \not\sq A_{2k}$. Let $\sigma: H \to \Z_2$ be the sign map. Then, $\sigma\varphi: G \to \Z_2$ is surjective and if $K = \ker(\sigma\varphi)$,
        \[G/K \cong \Z_2.\]
    That is, $K$ is a normal subgroup of index 2.
\end{proof}

\begin{prob}{F16.G1}{F16.G1}
    \begin{enumerate}[(i)]
        \item Suppose that $G$ has no subgroup of index 2. Prove that any subgroup of index 3 is normal.
        \item Let $G$ be the finite simple group of order 168. Prove that $G$ is a subgroup of $A_8$.
    \end{enumerate}
\end{prob}

\begin{proof}
    Let $H \sq G$ be of index 3. Let $\varphi: G \to S_3$ be the homomorphism of the group action of $G$ acting on the left cosets of $H$ by $g\cdot xH = (gx)H$. Notice that $\varphi$ is therefore a nontrivial homomorphism. Define $K = \ker(\varphi)$ so that $K$ is a normal subgroup of $G$. We prove that $K = H$ to show that $H$ is normal in $G$.\\

    If $a \in K$, then $H = a\cdot H = aH$ implying that $a \in H$. That is, $K \sq H$ and so $[G:K] \geq [G:H]$. By the First Isomorphism Theorem, $G/K$ is isomorphic to some subgroup of $S_3$. Since $[G:K] = |G/K|$ which must divide 6 and $[G:K] \geq [G:H] = 3$, the only possible values for $[G:K]$ are 3 or 6. If $[G:K] =6$, then $G/K \cong S_3$ which has a normal subgroup of index 2. Then $G/K$ must have a subgroup, say $J/K$, of index 2. This would then imply that $[G:J] =2$, a contradiction. Therefore, $[G:K] = 3$ and so $H = K$.
\end{proof}

\begin{proof}
    Notice that $168 = 2^3 \cdot 3 \cdot 7$. By the Sylow Theorems, $n_7 = 1$ or $n_7 = 8$. Since $G$ is simple, $n_7 \neq 1$ or else $G$ would have a nontrivial normal subgroup. Therefore $n_7 = 8$. Map $G$ into $S_8$ by letting $G$ act on the eight Sylow-7 subgroups via conjugation. Let $\varphi: G \to S_8$ be the homomorphism associated with this action. Let $\rm{sgn}: S_8 \to \Z_2$ be the sign map and let $\psi = \rm{sgn}\circ \varphi: G \to \Z_2$. Suppose that $\varphi(G) \not\sq A_8$. Then there exists some odd permutation in $\varphi(G)$ and therefore $\psi$ is a surjection. By the First Isomorphism Theorem, this implies that $G/\ker(\psi) \cong \Z_2$. This is impossible since then $\ker(\psi)$ would be a nontrivial normal subgroup of $G$. Therefore every element in $\varphi(G)$ must be an even permutation. Again, because $G$ has no nontrivial normal subgroups, the kernel of $\varphi$ must be trivial meaning that $\varphi$ is indeed an embedding.
\end{proof}

\end{document}