\documentclass[../AlgebraQualSolutions.tex]{subfiles}

\begin{document}

\subsection{Permutation Groups}

\begin{prob}{S19.G2}{S19.G2}
    You may assume that every permutation has a cycle decomposition $\sigma = \sigma_1 \cdots \sigma_k$ where $\sigma_1, \ldots, \sigma_k$ are pairwise disjoint cycles. Prove that $\sigma, \tau \in S_n$ are conjugate in $S_n$ if and only if for each $m \geq 2$ the number of $m$-cycles in a cycle decomposition of $\sigma$ equals the number of $m$-cycles in a cycle decomposition of $\tau$.
\end{prob}

\begin{proof}
\begin{claim}

    Two cycles $\sigma, \tau \in S_n$ are conjugate if and only if they are cycles of the same length.
    \begin{proof}
        Suppose first that $\tau = x \sigma x\inv$ for some $x \in S_n$. Further suppose that $\sigma$ is an $m$-cycle and $\tau$ is a $k$-cycle. Since $\sigma$ is an $m$-cycle, $\sigma^m = \cycle{1}$. Observe that
            \[\tau^m = (x\sigma x\inv)^m = x\sigma^m x\inv = xx\inv = \cycle{1}.\]
        This implies that the order of $\tau$ is at most $m$. Suppose now that $\tau^k = \cycle{1}$ for some $k < m$. Then, 
            \[\cycle{1} = \tau^k = (x\sigma x\inv)^k = x \sigma^k x\inv.\]
        Left-multiplying by $x\inv$ and right-multiplying by $x$ implies that $\sigma^k = \cycle{1}$, contradicting the minimality of $m$. Therefore, but $\sigma$ and $\tau$ are $m$-cycles.\\
        
        Now assume that $\sigma$ and $\tau$ are both $m$-cycles. Write $\sigma = \cycle{k_1, \ldots, k_m}$ and $\tau = \cycle{\ell_1, \ldots, \ell_m}$ where $k_1,\ldots,k_m \in \{1,\ldots, n\}$ are distinct and $\ell_1, \ldots, \ell_m \in \{1,\ldots, n\}$ are  distinct. Let $k_1', \ldots, k_{n-m}'$ and $\ell_1', \ldots, \ell_{n-m}'$ denote the elements of $\{1,\ldots, n\}$ that do not appear in the cycle notation for $\sigma$ and $\tau$, respectively.\\

        Define an element $x \in S_n$ as follows: for each $k_i$, let $x: k_i \mapsto \ell_i$. Note that defining $x$ in this way is well-defined as each $k_i$ is distinct. Similarly, for each $k_i'$, let $x: k_i' \mapsto \ell_i'$. It remains to show that $x\sigma x \inv = \tau$.\\

        Consider any $\ell_i$. By construction,
            \[x\sigma x\inv(\ell_i) = x\sigma(k_i) = x(k_{i+1}).\]
        If $i = m$, then this equality becomes
            \[x \sigma x\inv(\ell_m) = x(k_1).\]
        On the other hand,
            \[\tau(\ell_i) = \ell_{i+1} = x(k_{i+1})\]
        and when $i = m$,
            \[\tau(\ell_m) = \ell_1 = x(k_1).\]
        Note also that $\tau$ fixes each $\ell_i'$. For each $\ell_i'$,
            \[x\sigma x\inv(\ell_i') = x\sigma(k_i') = x(k_i') = \ell_i'\]
        since $\sigma$ fixes each $k_i'$. Adjust the equation as done previously in the case that $i = m$. This proves that $x\sigma x\inv = \tau$.
    \end{proof}
\end{claim}
\end{proof}


\begin{prob}{F17.G1}{F17.G1}
    \begin{enumerate}[(i)]
        \item Let $G$ be a finite group and $H \leq G$ such that $[G:H] = n$. Prove that there is a nontrivial homomorphism $\rho: G \to S_n$ with $\ker(\rho)\leq H$.
        \item Let $G$ be a finite group and $H$ a subgroup of $G$ whose index in $G$ is the smallest prime dividing $|G|$. Show that $H$ is a normal subgroup of $G$.
    \end{enumerate}
\end{prob}

\begin{proof}
    Let $\rho: G \to S_n$ be the homomorphism given by $G$ acting on the left cosets of $H$ in $G$ via multiplication. If $g \in \ker(\rho)$ then $g\cdot(xH) = H$ for every left coset $xH$. In particular, $g\cdot(H) = H$ and therefore $g \in H$. That is, $\ker(\rho) \leq H$.\\

    Since $[G:H] > 1$, there exists some $g \in G\sm H$. Then, $g\cdot H = gH \neq H$ and so $\rho$ is nontrivial.
\end{proof}

\begin{proof}
    Suppose that $p$ is the smallest prime dividing $|G|$ and that $H \leq G$ is a subgroup such that $[G:H] = p$. From the previous proof, there exists a nontrivial homomorphism $\rho: G \to S_p$ such that $\ker(\rho) \sq H$. Define $K = \ker(\rho)$. By the First Isomorphism Theorem, $G/K$ is isomorphic to some subgroup $J$ of $S_p$. Then $|J|$ must divide $|S_p| = p!$ and $|J|$ must divide $|G|$. If $|G/K| = |J| = 1$, then $G = K$. Then $H = G$ and is thus trivially normal in $G$. Otherwise, the minimality of $p$ implies that $|J| = p$. Then $[G:K] = p = [G:H]$ and since $K \leq H$, it follows  that $H = K$, a normal subgroiup.
    
    Since $p$ is the smallest prime that divides 
\end{proof}


\begin{prob}{S18.G1}{S18.G1}
    Show that if $G$ is a group of order $2k$ where $k$ is odd, then $G$ has a subgroup of index 2.
\end{prob}

\begin{proof}
    Enumerate the elements of $G$ with values $\{1,\ldots, 2k\}$. For each $a \in G$, let $\varphi_a: G \to G$ be left multiplication by $a$. Identify $\varphi_a$ with the permutation in $S_{2k}$ corresponding to the image of $\varphi_a$. Then define $\varphi: G \to S_{2k}$ by $\varphi: a \mapsto \varphi_{a} \in S_{2k}$. By construction, $\varphi$ is injective and a homomorphism meaning that $G$ is isomorphic to some subgroup $H$ of $S_{2k}$.\\

    By Cauchy's Theorem (or \ref{prob:F12.G1}), $G$ has an element $x$ of order 2 since the order of $G$ is divisible by two. Consider $\varphi(x) \in H$. Since $x$ is of order 2 and $\varphi$ is an isomorphism, $\varphi(x)$ must also be of order 2. That is, $\varphi(x)$ is the product of some number of disjoint transpositions. Since $|G| = 2k$, there are at most $k$ disjoint transpositions in $\varphi(x)$. If $\varphi(x)$ is the product of less than $k$ transpositions, then there exists $g \in G$ such that $xg = g$. However, this would imply that $x = e$ which contradicts the fact that $x$ is of order 2. Therefore $\varphi(x)$ is the product of exactly $k$ disjoint transpositions. Since $k$ is odd, $H \not\sq A_{2k}$. Let $\sigma: H \to \Z_2$ be the sign map. Then, $\sigma\varphi: G \to \Z_2$ is surjective and if $K = \ker(\sigma\varphi)$,
        \[G/K \cong \Z_2.\]
    That is, $K$ is a normal subgroup of index 2.
\end{proof}

\begin{prob}{F16.G1}{F16.G1}
    \begin{enumerate}[(i)]
        \item Suppose that $G$ has no subgroup of index 2. Prove that any subgroup of index 3 is normal.
        \item Let $G$ be the finite simple group of order 168. Prove that $G$ is a subgroup of $A_8$.
    \end{enumerate}
\end{prob}

\begin{proof}
    Let $H \sq G$ be of index 3. Let $\varphi: G \to S_3$ be the homomorphism of the group action of $G$ acting on the left cosets of $H$ by $g\cdot xH = (gx)H$. Notice that $\varphi$ is therefore a nontrivial homomorphism. Define $K = \ker(\varphi)$ so that $K$ is a normal subgroup of $G$. We prove that $K = H$ to show that $H$ is normal in $G$.\\

    If $a \in K$, then $H = a\cdot H = aH$ implying that $a \in H$. That is, $K \sq H$ and so $[G:K] \geq [G:H]$. By the First Isomorphism Theorem, $G/K$ is isomorphic to some subgroup of $S_3$. Since $[G:K] = |G/K|$ which must divide 6 and $[G:K] \geq [G:H] = 3$, the only possible values for $[G:K]$ are 3 or 6. If $[G:K] =6$, then $G/K \cong S_3$ which has a normal subgroup of index 2. Then $G/K$ must have a subgroup, say $J/K$, of index 2. This would then imply that $[G:J] =2$, a contradiction. Therefore, $[G:K] = 3$ and so $H = K$.
\end{proof}

\begin{proof}
    Notice that $168 = 2^3 \cdot 3 \cdot 7$. By the Sylow Theorems, $n_7 = 1$ or $n_7 = 8$. Since $G$ is simple, $n_7 \neq 1$ or else $G$ would have a nontrivial normal subgroup. Therefore $n_7 = 8$. Map $G$ into $S_8$ by letting $G$ act on the eight Sylow-7 subgroups via conjugation. Let $\varphi: G \to S_8$ be the homomorphism associated with this action. Let $\rm{sgn}: S_8 \to \Z_2$ be the sign map and let $\psi = \rm{sgn}\circ \varphi: G \to \Z_2$. Suppose that $\varphi(G) \not\sq A_8$. Then there exists some odd permutation in $\varphi(G)$ and therefore $\psi$ is a surjection. By the First Isomorphism Theorem, this implies that $G/\ker(\psi) \cong \Z_2$. This is impossible since then $\ker(\psi)$ would be a nontrivial normal subgroup of $G$. Therefore every element in $\varphi(G)$ must be an even permutation. Again, because $G$ has no nontrivial normal subgroups, the kernel of $\varphi$ must be trivial meaning that $\varphi$ is indeed an embedding.
\end{proof}

\end{document}