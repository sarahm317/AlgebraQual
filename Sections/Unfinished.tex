\documentclass[../AlgebraQualSolutions.tex]{subfiles}

\begin{document}
\section{Unfinished}

%%%%%%%%%%%%%%%%%%%%%%%%%%%%%%%%%%%%%%%%%%%%%%%%%%%%%%%%%%%%%%%%%%%%%%%%%%%%%%%%%%%%%%%%%%%%%%%%%%%%%%%%%%

\subsection{Spring 2013}

\begin{prob}{S13}{S13.G1}
	Fix a prime $p$ and let $A, A_1, A_2,$ and $B$ be finite abelian $p$-groups.

	\begin{enumerate}[(a)]
		\item Assume that $A$ and $B$ are cyclic and that the number of elements of order at most $p^r$ in $A$ equals the corresponding number in $B$ for each $r \in \N$. Show that $A \cong B$.
		\item Suppose that $A = A_1 \oplus A_2$. Show that for any $r \in \N$ the number of elements of order at most $p^r$ in $A$ equals $n_1\cdot n_2$ where $n_i$ is the number of elements of order at most $p^r$ in $A_i$.
		\item Prove that $A$ is isomorphic to $B$ if for each $r\in \N$ the number of elements of order at most $p^r$ in $A$ equals the number of elements of order at most $p^r$ in $B$. \emph{You may use without proof the Fundamental Theorem for Finite Abelian Groups. What does the number of elements of order at most $p$ in $A$ tell you about the number of nontrivial cyclic direct summands of $A$?}
	\end{enumerate}
\end{prob}

\begin{prob}{S13}{S13.G2}

	Let  $X = \{1,\ldots, n\}$. For any $\tau \in S_n$, the \emph{support} of $\tau$ is the set $\{i \in X: \tau(i) \neq i\}$. Let $\sigma \in S_n$ be nontrivial and consider the equivalence relation on $X$ given by $a \sim b$ if and only if there exists $m \in \Z$ where $\sigma^m(a) = b$.

	\begin{enumerate}[(a)]
		\item Let $X_0 \sq X$ be some equivalence class. Show that the restriction of $\sigma$ to $X_0$ is a cyclic permutation of $X_0$.
		\item Prove that $\sigma$ is a product of cycles $\sigma_1,\ldots,\sigma_r$ with pairwise disjoint supports.
		\item Show that the order of $\sigma$ in $S_n$ equals the least common multiple of the cardinalities of the supports of the $\sigma_i$.
	\end{enumerate}
\end{prob}

\begin{prob}{S13}{S13.G3}
	Let $G$ be a finite group and $H$ a proper subgroup of $G$ such that $|G|$ does not divide $[G:H]!$. Prove that $H$ contains a nontrivial normal subgroup of $G$.
\end{prob}

\begin{prob}{S13}{S13.RF1}

	\begin{enumerate}[(a)]
		\item Partition the following algebras over $\Q$ into classes of pairwise disjoint isomorphic algebras. Provide full reasoning.
			\[M_3(\Q) \quad  \quad \Q[x]/(x^9 - 1) \quad \quad \Q[x]/(x^9 + 6x^2 - 3) \quad  \quad \Q[x]/(x-1) \times \Q[x]/(x^8 + x^7 + \cdots + x + 1)\]
		\item Which of the algebras above are fields?
	\end{enumerate}
\end{prob}

\begin{prob}{S13}{S13.RF2}
	Prove the following case of Eisenstein's irreducibility criterion: Suppose that $f = x^n + \sum_{i=0}^{n-1} a_i x^i$ is a polynomial in $\Z[x]$ of positive degree $n$ and $p \in \N$ is a prime such that $p$ divides $a_i$ for $0 \leq i \leq n - 1$ but $p^2$ does not divide $a_0$. Then, $f$ is irreducible over $\Z$.
\end{prob}

\begin{prob}{S13}{S13.RF3}
Give examples of polynomials in $\Q[x]$ whose splitting fields over $\Q$ have Galois groups over $\Q$ as specified:
	\begin{enumerate}[(a)]
		\item The Galois group is isomorphic to $S_3$.
		\item The Galois group is cyclic of order 6.
	\end{enumerate}
\end{prob}

\begin{prob}{S13}{S13.LA1}
Let $A$ be an $n \times n$ matrix over $\C$.
	\begin{enumerate}[(a)]
		\item State the theorem addressing existence and (qualified) uniqueness of a Jordan canonical form of $A$.
		\item Show that $A$ is nilpotent if and only if all eigenvalues of $A$ are zero.
		\item Give all possible similarity classes of matrices $A \in M_6(\C)$ in Jordan form which satisfy the condition that  $A^4 = 0$ but $A^3 \neq 0$.
	\end{enumerate}
\end{prob}

\begin{prob}{S13}{S13.LA2}

	\begin{enumerate}[(a)]
		\item Suppose that $V$ is an $n$-dimensional vector space over a field $F$ and $\{b_1, \ldots, b_n\}$ is a basis for $V$. Let $T: V \to V$ be a linear transformation. Show that $T$ is an isomorphism if and only if the set $\{T(b_1), \ldots, T(b_n)\}$ is a basis for $V$.
		\item Let $A \in M_n(F)$. Deduce from (a) that $A$ is invertible if and only if the columns of $A$ form a basis for the space $F^n$  of $n \times 1$ column vectors over $F$.
		\item Given a finite field $F$ with $q$ elements, determine the order of the group $GL_2(F)$ of all invertible $2 \times 2$ matrices over $F$.
	\end{enumerate}
\end{prob}

\begin{prob}{S13}{S13.LA3}
The group $S_3$ has exactly three conjugacy classes: $\{(1)\}$, $\{(1,2,3), (1,3,2)\}$, and the set of transpositions. Let $V$ be the vector space over $\C$ consisting of all functions $f:S_3 \to \C$ which are constant on the conjugacy classes.
	\begin{enumerate}[(a)]
		\item Show that $\ip{\cdot}{\cdot}: V \times V \to \C$ given by 
			\[\ip{f}{g} = \frac{1}{6} \sum_{x,y \in S_3} f(x) \ol{g(y)}\]
		is a Hermitian inner product on $V$. Verify that $f_1 = 1$ and $f_2 = \rm{sgn}$ in $V$ are orthogonal and each of norm 1. 
	\item Extend $\{f_1, f_2\}$ to an orthonormal basis for $V$.
	\end{enumerate}
\end{prob}

%%%%%%%%%%%%%%%%%%%%%%%%%%%%%%%%%%%%%%%%%%%%%%%%%%%%%%%%%%%%%%%%%%%%%%%%%%%%%%%%%%%%%%%%%%%%%%%%%%%%%%%%%%

\subsection{Fall 2012}

\begin{prob}{F12}{F12.RF1}
\begin{enumerate}[(a)]
\item Let $\nu$ be a primitive 10th root of unity. Compute the Galois group of $\Q(\nu)$.
\item let $K_t = \Q(\omega)$ where $\omega$ is a primitive $t$th root of unity. Show that $\sqrt[3]{2}$ is not in any $K_t$.
\end{enumerate}
\end{prob}

\begin{prob}{F12}{F12.RF2}
Suppose that $R,S,A$ are all commutative rings with unity.
\begin{enumerate}[(a)]
\item Suppose that $R$ is a PID, $S$ is an integral domain, and $\varphi:R \to S$ is a surjection. Prove that either $\varphi$ is an isomorphism or $S$ is a field.
\item Prove that $A[x]$ is a PID if and only if $A$ is a field.
\end{enumerate}
\end{prob}

\begin{prob}{F12}{F12.LA3}
Let $M$ be a $5 \times 5$ matrix with rational entries whose characteristic polynomial is of the form
	\[(x^2+1)(x^3+x+q).\]
For which, if any, $q \in \Q$ is it possible that there does not exist an invertible matrix $A$ with rational entries such that $A\inv M A$ has block diagonal form with a $2 \times 2$ block and a $3 \times 3$ block?
\end{prob}

%%%%%%%%%%%%%%%%%%%%%%%%%%%%%%%%%%%%%%%%%%%%%%%%%%%%%%%%%%%%%%%%%%%%%%%%%%%%%%%%%%%%%%%%%%%%%%%%%%%%%%%%%%
\subsection{Spring 2012}

\begin{prob}{S12}{S12.G3}
	\begin{enumerate}[(a)]
		\item State the Sylow Theorems.
		\item Prove that every group of order 126 has a normal subgroup of order 7.
		\item Prove that any group of order 1000 is not simple.
	\end{enumerate}
\end{prob}

\begin{prob}{S12}{S12.RF1}
	Prove or disprove:

	\begin{enumerate}[(a)]
		\item If $R$ is an integral domain, then $R[x]$ is an integral domain.
		\item If $R$ is a principal ideal domain, then $R[x]$ is a principal ideal domain.
	\end{enumerate}
\end{prob}

\begin{prob}{S12}{S12.RF2}
	Let $F$ be a finite field.

	\begin{enumerate}[(a)]
		\item Show that the multiplicative group $F^*$ is cyclic.
		\item Suppose $|F| = 125 = 5^3$ and $\cyc{\alpha} = F^*$. What is $\alpha^{62}$?
		\item Is there a $\beta \neq \alpha$ in $F$ such that $\cyc{\beta} = F^*$?
	\end{enumerate}
\end{prob}

\begin{proof}
	This is a subset of the problem \ref{prob:S20.RF2}. The other problem shows that any finite subgroup of the multiplicative group of a field is cyclic (not just when the field itself is finite).
\end{proof}

\begin{prob}{S12}{S12.RF3}
	Let $\alpha = \sqrt{-1+\sqrt{2}}$.
	\begin{enumerate}[(a)]
		\item Prove that $\alpha$ is the root of a monic polynomial in $\Q[x]$.
		\item Let $K$ be the smallest Galois extension of $\Q$ that contains $\alpha$. Find the degree $[K: \Q]$.
	\end{enumerate}
\end{prob}

\begin{prob}{S12}{S12.LA1}

	Consider a real vector space $V = \R^n$ with the Euclidean inner product and let $U$ be a subspace of $V$.

	\begin{enumerate}[(a)]
		\item Prove that $U$ has an orthonormal basis. Note that this is the real version of the Gram-Schmidt Theorem.
		\item Find an orthonormal basis for the span of $(1,2,0)$ and $(1,1,3)$ in $\R^3$.
	\end{enumerate}
\end{prob}

\begin{prob}{S12}{S12.LA3}
	Let $V$ be a finite dimensional vector space over $\C$.

	\begin{enumerate}[(a)]
		\item Define the characteristic polynomial of a linear transformation of $V$ and the minimal polynomial of a linear transformation of $V$.
		\item Give an example of two linear transformations $S,T: V \to V$ such that $S$ and $T$ have the same characteristic polynomial, but are not similar.
		\item Give an example of two linear transformations $S,T: V \to V$ such that $S$ and $T$ have the same minimal polynomial, but are not similar.
	\end{enumerate}
\end{prob}

The characteristic polynomial of a linear transformation $T: V \to V$ is $(x - \lambda_1)^{\ell_1}\cdots(x-\lambda_m)^{\ell_m}$ where $\lambda_1,\ldots, \lambda_j$ are the distinct eigenvalues of $T$ and $\ell_j$ is the multiplicity of the eigenvalue $\lambda_j$. The minimal polynomial of $T$ is the smallest degree monic polynomial $p$ such that $p(T) = 0$.\\

The minimal polynomial and characteristic polynomial have the same roots, ignoring multiplicity. The minimal polynomial divides the characteristic polynomial.

%%%%%%%%%%%%%%%%%%%%%%%%%%%%%%%%%%%%%%%%%%%%%%%%%%%%%%%%%%%%%%%%%%%%%%%%%%%%%%%%%%%%%%%%%%%%%%%%%%%%%%%%%%
\subsection{Fall 2013}

\begin{prob}{F13}{F13.G2}
	Prove that every finite group of order greater than two has a non-trivial automorphism.
\end{prob}

\begin{proof}
Suppose that $G$ is a finite group of order greater than two. Fix some non-identity element $x \in G$. Define $\varphi: G \to G$ by $\varphi: y \mapsto xyx\inv$. Notice that for any $a,b \in G$,
	\[\varphi(ab) = xabx\inv = (xax\inv)(xbx\inv) = \varphi(a)\varphi(b) \]
proving that $\varphi$ is a homomorphism. Observe that $\varphi$ is bijective with inverse given by $y \mapsto x\inv y x$. Therefore, $\varphi$ is an automorphism.
\end{proof}

\begin{prob}{F13}{F13.G3}

	\begin{enumerate}[(a)]
		\item Let $G$ be a finite group and $H$ a subgroup of $G$. Prove that the order of $H$ divides the order of $G$.
		\item  Let $\Q$ and $\Z$ denote the additive groups of the rationals and integers, respectively. Prove that $\Q/\Z$ has no proper subgroups of finite index.
	\end{enumerate}
\end{prob}

\begin{prob}{F13}{F13.RF1}
	Let $V = \R^2$, regarded as a two-dimensional subspace over $\R$. Let $L(V)$ be the ring of all linear transformations from $V$ to $V$. let $T \in L(V)$ be given by $T(x,y) = (y,-x)$ and define 
		\[A = \{S \in L(V): ST = TS\}. \]
	\begin{enumerate}[(a)]
		\item Prove that $A$ is a subring of $L(V)$.
		\item To what well-known ring is $A$ isomorphic?
	\end{enumerate}
\end{prob}

\begin{prob}{F13}{F13.RF2}

	\begin{enumerate}[(a)]
		\item Let $N$ be a non-negative integer and $\alpha \in \C$ a primitive $N$th root of unity. FOr which $N$ is it true that $\Q(\alpha) = (\alpha + \alpha\inv)$?
		\item Let $K$ be a field and $\beta$ an element of the algebraic closure of $K$. If $[K(3):K]$ is odd, prove that $K(\beta) = K(\beta + \beta\inv)$.
	\end{enumerate}
\end{prob}

\begin{prob}{F13}{F13.RF3}
	Let $F$ be a field and $p_1(x), \ldots, p_r(x)$ distinct, monic, irreducible polynomials in $F[x]$. Let $f(x) = p_1(x)^{n_1} + \cdots + p_r(x)^{n_r}$ where each $n_i$ is a positive integer.

	\begin{enumerate}[(a)]
		\item Determine the number of ideals in $F[x]/\cyc{f(x)}$.
		\item Determine the number of prime ideals in $F[x]/\cyc{f(x)}$.
	\end{enumerate}
\end{prob}

\begin{prob}{F13}{F13.LA1}
	Let $T: \R^3 \to \R^3$ be a linear transformation. Prove the following:
	\begin{enumerate}[(a)]
		\item $T$ has a one-dimensional invariant subspace.
		\item $T$ has a two-dimensional invariant subspace.
	\end{enumerate}

	\emph{ Do not use the existence of canonical forms for matrices.}
\end{prob}

\begin{prob}{F13}{F13.LA2}
	Let $V$ be a finite dimensional vector space of $\Q$ and let $M$ be an automorphism of $V$ such that $M$ fixes no non-zero vector in $V$. Suppose that $M^p$ is the identity map on $V$ with $p$ a prime. Show that the dimension of $V$ is divisible by $p-1$. \emph{ You may assume that the polynomial $x^{p-1} + \cdots + x + 1$ is irreducible  over $\Q$.}
\end{prob}

\begin{prob}{F13}{F13.LA3}
	Let $\R^3$ have  the usual inner product and suppose that $(a,b,c) \in \R^3$ is of unit length. Let $W$ be the plane given by $ax + by + cz = 0$. Let $\ell$ be the line through the origin in the direction $(a,b,c)$.
	\begin{enumerate}[(a)]
		\item Define the standard matrix representing the orthogonal projection of $\R^3$ onto $\ell$.
		\item Define the standard matrix representing the orthogonal projection of $\R^3$ onto $W$.
	\end{enumerate}
\end{prob}
%%%%%%%%%%%%%%%%%%%%%%%%%%%%%%%%%%%%%%%%%%%%%%%%%%%%%%%%%%%%%%%%%%%%%%%%%%%%%%%%%%%%%%%%%%%%%%%%%%%%%%%%%%	

\begin{prob}{F19}{}
Let $p$ be a prime number and suppose that $1 \leq n <p^2$ is an integer. Show that every Sylow $p$-subgroup of the symmetric group $S_n$ is abelian.
\end{prob}


\begin{prob}{F20}{F20.G3}
Let $G$ be a finite group, $x \in G$, and $H$ a subgroup of $G$.
\begin{enumerate}[(a)]
\item Prove that the number of conjugates of $x$ in $G$ divides $\rm{order}(G)/\rm{order}(x)$.
\item Prove that the number of conjugates of $H$ in $G$ divides the index of $H$ in $G$.
\end{enumerate}
\end{prob}

\begin{prob}{S19}{}
State and prove Lagrange's Theorem. Prove that a subgroup of a cyclic group is cyclic.
\end{prob}

\begin{prob}{S19}{}
Prove that $\sigma,\tau \in S_n$ are conjugate if and only if for each $m \geq 2$ the number of $m$-cycles in a cycle decomposition of $\sigma$ equals the number of $m$-cycles in a cycle decomposition of $\tau$.
\end{prob}

\begin{prob}{F20}{F20.G1}
Show that any finitely generated group of $(\Q,+)$ is cyclic. Use this to prove that the direct product $(\Q,+) \times (\Q,+)$ and $(\Q,+)$ are not isomorphic.
\end{prob}

\begin{prob}{S20}{}
Let $R$ be a commutative ring with identity. Show that if $p$ is a prime ideal in $R$ then
	\[p(x) = \left\{\textrm{all polynomials } \sum_i a_ix^i \textrm{ with each } a_i \in p \right\} \]
is a prime ideal in $R[x]$.
\end{prob}

\begin{prob}{F20}{F20.RF2}
Prove or provide a counter example: Suppose that $K$ is a finite extension of $F$. $F \sq L \sq K$, $F \sq M \sq KL$, $LM = K$ and $L \cap M = F$. Then $[L:F][M:F] = [K:F]$. Here, $LM$ denotes the composition of the fields $L$ and $M$.
\end{prob}

\begin{prob}{F20}{F20.RF3}
	\begin{enumerate}[(i)]
	\item Let $R$ be a UFD and $d$ a nonzero element in $R$. Prove that there are only finitely many principal ideals  in $R$ that contain the ideal $(d)$.
	\item Give an example of a UFD $R$ and a nonzero element $d \in R$ such that there are infinitely many ideals in $R$ that contain $(d)$.
	\end{enumerate}
\end{prob}

\begin{prob}{F20}{F20.LA1}
It is known that real symmetric matrices are always diagonalizable. You may assume this fact.
\begin{enumerate}[(a)]
\item What special property do the eigenspaces of a real symmetric matrix have?
\item Prove that any real symmetric matrix $S$ can be diagonalized by an orthonormal matrix $U$.
\end{enumerate}
\end{prob}

\begin{prob}{F20}{F20.LA3}
Let $V$ be a finite dimensional complex vector space and $T: V \to V$ a linear transformation.
\begin{enumerate}[(a)]
\item Show that $V$ has a "flag" of subspaces $V_0 = 0 \sq V_1 \sq \cdots \sq V_n = V$ such that $\dim(V_i) = i$ and $T(V_i) \sq V_i$ for each $i$.
\item Show that there is a basis for $V$ such that the matrix of $T$ with respect to this basis is upper triangular.
\end{enumerate}
\end{prob}

\begin{proof}
	See \ref{thm:upper}.
\end{proof}

\end{document}