\documentclass[../AlgebraQualSolutions.tex]{subfiles}

\begin{document}
\section{Unfinished}
	
	\begin{prob}{F12}{F12.G2}
	Define the center of a group.
	\begin{enumerate}[(a)]
	\item Prove that if the order of $G$ is $p^k$ for some prime $p$ then $G$ has nontrivial center.
	\item Suppose that $p$ and $q$ are distinct primes. Prove that a non-abelian group of order $pq$ has trivial center.
	\end{enumerate}
	\end{prob}
	
	The center of a group $G$ is the subgroup
		\[Z(G) = \{g \in G: gx = xg \textrm{ for any $x \in G$} \} \]
	
	\begin{proof}
	Assume that $|G| = p^k$ for some prime $p$. Suppose that $Z(G)$ is trivial and consists only of the identity element $e \in G$. If $k = 1$, the order of $G$ is prime which implies that $G$ is cyclic and therefore abelian. Therefore, assume that $k > 1$.
	\end{proof}
	
	
	\begin{prob}{F12}{F12.G3}
	\begin{enumerate}[(a)]
	\item State and prove Lagrange's Theorem.
	\item Prove that a subgroup of a cyclic group is cyclic.
	\end{enumerate}
	\end{prob}
	
	Lagrange's Theorem States the following: If $G$ is a finite group and $H$ is a subgroup of $G$ then $[G:H]\cdot|H| = |G|$ where $[G:H]$ is the number of distinct left cosets of $H$ in $G$.

	\begin{proof}
		Suppose that $G$ is a finite group and $H \leq G$ is a subgroup.

		\begin{claim}
			The set of left cosets of $H$ in $G$ partitions $G$.

			\begin{proof}
				Note that any $g \in G$ is in the coset $gH$ since $g = g\cdot e$ where $e \in H$ is the identity element of the group. That is, each element of $G$ is in some left coset of $H$.\\

				Let $a,b \in G$ and suppose that $aH \cap bH \neq \varnothing$. Let $y \in aH \cap bH$ with $y = ah_1 = bh_2$ where $h_1,h_2 \in H$. Observe that $a = bh_1h_2\inv$ and $b = ah_2h_1\inv$. Then, for any $h \in H$,
					\[ah = bh_1h_2\inv h \in bH \]
				and
					\[bh = ah_2h_1\inv h \in aH.\]
				Therefore, $aH = bH$. Since any pair of left cosets are either disjoint or equal, it follows that the set of left cosets partitions $G$.
			\end{proof}
		\end{claim}

		\begin{claim}
			For any $g \in G$, $|H| = |gH|$.

			\begin{proof}
				Define $\varphi: H \to gH$ by $\varphi(h) = gh$. Clearly $\varphi$ is surjective. If $gh = gh'$, multiplying each side by $g\inv$ implies that $h = h'$ and so $\varphi$ is injective. As $\varphi$ is a bijection between $H$ and $gH$, $|H| = |gH|$ as desired. 
			\end{proof}
		\end{claim}

		As $G$ is finite, there are a finite number of left cosets of $H$ in $G$. Let $g_1H, \ldots, g_nH$ be the left cosets of $H$ in $G$. Each $g_iH$ has $|H|$ elements and each $g \in G$ is in some $g_iH$. Therefore,
			\[|G| = n|H| = [G:H]|H|,\]
		as desired.
	\end{proof}

	\begin{proof}
		Suppose that $G$ is a cylic group and $H \leq G$ a subgroup. Since $G$ is cyclic, there exists $g \in G$ such that $G = \langle g \rangle$. 
	\end{proof}
	
	\begin{prob}{F12}{F12.RF1}
	\begin{enumerate}[(a)]
	\item Let $\nu$ be a primitive 10th root of unity. Compute the Galois group of $\Q(\nu)$.
	\item let $K_t = \Q(\omega)$ where $\omega$ is a primitive $t$th root of unity. Show that $\sqrt[3]{2}$ is not in any $K_t$.
	\end{enumerate}
	\end{prob}
	
	\begin{prob}{F12}{F12.RF2}
	Suppose that $R,S,A$ are all commutative rings with unity.
	\begin{enumerate}[(a)]
	\item Suppose that $R$ is a PID, $S$ is an integral domain, and $\varphi:R \to S$ is a surjection. Prove that either $\varphi$ is an isomorphism or $S$ is a field.
	\item Prove that $A[x]$ is a PID if and only if $A$ is a field.
	\end{enumerate}
	\end{prob}
	
	\begin{prob}{F12}{F12.RF3}
	\begin{enumerate}[(a)]
	\item Prove that $\Z/m\Z$ has no non-zero nilpotent elements if and only if $m$ has no multiple prime factor.
	\item Prove that every element of $\Z/m\Z$ is either nilpotent or a unit whenever $m$ is a prime power.
	\item Prove that if $r$ is a nilpotent element of a ring with unity then $1-r$ is a unit.
	\end{enumerate}
	\end{prob}
	
	\begin{prob}{F12}{F12.LA1}
	Let $V$ be the vector space of real $n \times n$ matrices. Show that 	
		\[\langle A, B \rangle = n \tr(AB) - \tr(A)\tr(B) \]
	defines a symmetric bilinear form on $V$.
	\begin{enumerate}[(a)]
	\item Prove that $\langle,\rangle$ is singular.
	\item Prove that the restriction of $\langle,\rangle$ to the subspace $W$ of symmetric matrices with 0 trace is positive definite.
	\end{enumerate}
	\end{prob}

	\begin{prob}{F12}{F12.LA3}
	Let $M$ be a $5 \times 5$ matrix with rational entries whose characteristic polynomial is of the form
		\[(x^2+1)(x^3+x+q).\]
	For which, if any, $q \in \Q$ is it possible that there does not exist an invertible matrix $A$ with rational entries such that $A\inv M A$ has block diagonal form with a $2 \times 2$ block and a $3 \times 3$ block?
	\end{prob}
	
	\begin{prob}{F19}{}
	Let $p$ be a prime number and suppose that $1 \leq n <p^2$ is an integer. Show that every Sylow $p$-subgroup of the symmetric group $S_n$ is abelian.
	\end{prob}
	

	\begin{prob}{F20}{F20.G3}
	Let $G$ be a finite group, $x \in G$, and $H$ a subgroup of $G$.
	\begin{enumerate}[(a)]
	\item Prove that the number of conjugates of $x$ in $G$ divides $\rm{order}(G)/\rm{order}(x)$.
	\item Prove that the number of conjugates of $H$ in $G$ divides the index of $H$ in $G$.
	\end{enumerate}
	\end{prob}

	\begin{prob}{S20}{}
	Let $K$ be a field. Show that every finite subgroup of $K^\times$ is cyclic.
	\end{prob}

	\begin{prob}{S19}{}
	State and prove Lagrange's Theorem. Prove that a subgroup of a cyclic group is cyclic.
	\end{prob}

	\begin{prob}{S19}{}
	Prove the following:
	\begin{enumerate}[(a)]
	\item If $R$ is a commutative ring with no nilpotent elements, then $R[x]$ has no nilpotent elements.
	\item If $r$ is a nilpotent element of a ring with unity, then $1 - r$ is a unit.
	\end{enumerate}
	\end{prob}
	

	
	\begin{prob}{S19}{}
	Prove that $\sigma,\tau \in S_n$ are conjugate if and only if for each $m \geq 2$ the number of $m$-cycles in a cycle decomposition of $\sigma$ equals the number of $m$-cycles in a cycle decomposition of $\tau$.
	\end{prob}
	
	\begin{prob}{S20}{}
	\begin{enumerate}[(a)]
	\item Prove that the centralizer of an element is a subgroup.
	\item If $G$ is a finite group, prove that the number of elements in the conjugacy class divides the order of $G$.
	\end{enumerate}
	\end{prob}
	
	\begin{prob}{F20}{F20.G1}
	Show that any finitely generated group of $(\Q,+)$ is cyclic. Use this to prove that the direct product $(\Q,+) \times (\Q,+)$ and $(\Q,+)$ are not isomorphic.
	\end{prob}

	\begin{prob}{S20}{}
	Let $R$ be a commutative ring with identity. Show that if $p$ is a prime ideal in $R$ then
		\[p(x) = \left\{\textrm{all polynomials } \sum_i a_ix^i \textrm{ with each } a_i \in p \right\} \]
	is a prime ideal in $R[x]$.
	\end{prob}
	
	\begin{prob}{S20}{}
	Let $\F_3$ be the field with 3 elements.
	\begin{enumerate}[(a)]
	\item Prove that $K = \F_3[x]/(x^2 + 1)$ is a field.
	\item How many elements does $K$ have?
	\item Prove that $x+1$ generates the multiplicative group of non-zero elements in $K$.
	\end{enumerate}
	\end{prob}

	
	\begin{prob}{F20}{F20.RF1}
	The polynomial $x^3 - x$ has six roots in the ring $\Z/6\Z$. Find a sufficient condition on a commutative ring $R$ which ensures that the number of roots of a polynomial with coefficients in $R$ cannot exceed its degree and justify your assertion.
	\end{prob}
	
	\begin{prob}{F20}{F20.RF2}
	Prove or provide a counter example: Suppose that $K$ is a finite extension of $F$. $F \sq L \sq K$, $F \sq M \sq KL$, $LM = K$ and $L \cap M = F$. Then $[L:F][M:F] = [K:F]$. Here, $LM$ denotes the composition of the fields $L$ and $M$.
	\end{prob}
	
	\begin{prob}{F20}{F20.RF3}
		\begin{enumerate}[(i)]
		\item Let $R$ be a UFD and $d$ a nonzero element in $R$. Prove that there are only finitely many principal ideals  in $R$ that contain the ideal $(d)$.
		\item Give an example of a UFD $R$ and a nonzero element $d \in R$ such that there are infinitely many ideals in $R$ that contain $(d)$.
		\end{enumerate}
	\end{prob}
	
	\begin{prob}{F20}{F20.LA1}
	It is known that real symmetric matrices are always diagonalizable. You may assume this fact.
	\begin{enumerate}[(a)]
	\item What special property do the eigenspaces of a real symmetric matrix have?
	\item Prove that any real symmetric matrix $S$ can be diagonalized by an orthonormal matrix $U$.
	\end{enumerate}
	\end{prob}
	
	\begin{prob}{F20}{F20.LA2}
	Prove or give a counter example.
	\begin{enumerate}[(a)]
	\item If a $4 \times 4$ real matrix has characteristic polynomial $x^4 -1$ then its minimal polynomial cannot be $x^2 - 1$.
	\item Every $n \times n$ real matrix is similar over the reals to an upper triangular matrix.
	\end{enumerate}
	\end{prob}
	
	\begin{prob}{F20}{F20.LA3}
	Let $V$ be a finite dimensional complex vector space and $T: V \to V$ a linear transformation.
	\begin{enumerate}[(a)]
	\item Show that $V$ has a "flag" of subspaces $V_0 = 0 \sq V_1 \sq \cdots \sq V_n = V$ such that $\dim(V_i) = i$ and $T(V_i) \sq V_i$ for each $i$.
	\item Show that there is a basis for $V$ such that the matrix of $T$ with respect to this basis is upper triangular.
	\end{enumerate}
	\end{prob}
	

	
	\begin{prob}{F21}{F21.G3}
	Let $p$ and $q$ be distinct odd primes. Use the Sylow Theorems to show that every group of order $p^2q^2$ is not simple.
	\end{prob}	
	
	\begin{proof}
	Suppose that $|G| = p^2q^2$ and without loss of generality, assume that $p < q$. Note that each Sylow $p$-subgroup is of order $p^2$ and each Sylow $q$-subgroup is of order $q^2$. Furthermore, the number of Sylow-$p$ subgroups is $n_p = 1, q, q^2$ and the number of Sylow-$q$ subgroups is $n_q = 1, p^2$. If $n_q = 1$, then the Sylow-$q$ subgroup is normal in $G$ and thus $G$ is not simple. Suppose instead that $n_q = p^2$.\\
	
	If $Q_i, Q_j$ are any two distinct Sylow-$q$ subgroups, the intersection $Q_i \cap Q_j$ is either of size $1$ or $q$. If every pairwise intersection between the Sylow-$q$ subgroups is trivial, then the $p^2$ Sylow-$q$ subgroups account for $p^2q^2 - (p^2 -1)$ elements implying that $n_p = 1$. That is, the Sylow-$p$ subgroup is normal and therefore $G$ is not simple. Suppose that some Sylow-$q$ subgroups$Q_1, Q_2$ have intersection of size $q$. Let $N = Q_1 \cap Q_2$ and let $M$ be the subgroup of $G$ generated by $Q_1$ and $Q_2$. Then, $|M| > |Q_1| = q^2$ and $|M|$ divides $|G| = p^2q^2$. Therefore $|M| = pq^2$ or $|M| = p^2 q^2$. 
	
Note first that any group of prime squared order is abelian. Therefore both $Q_1$ and $Q_2$ are abelian and so $N = Q_1 \cap Q_2$ is normal in $M$. If $|M| = p^2q^2$ then $M=G$ and thus $N$ is a normal subgroup of $G$ and is nontrivial.\\

Now suppose that $|M| = pq^2$. 
	\end{proof}
	
	\begin{prob}{S19}{}
	\begin{enumerate}[(a)]
	\item Suppose that $G$ is a group and $G/Z(G)$ is cyclic. Prove  that $G$ is abelian.
	\item Let $p$ be a prime number and $G$ a non-cyclic finite $p$-group. Prove that $G$ contains a normal subgroup $N$ such that $G/N \cong C \oplus C$ where $C$ is a cyclic group of order $p$.
	\end{enumerate}
	\end{prob}
	
	\begin{proof}
	See \ref{prob:GZG}.
	\end{proof}
	
	\begin{proof}
	Suppose that $|G| = p^k$ with $k \geq 2$. Assume that $G$ is not cyclic.
	\end{proof}
	
\end{document}