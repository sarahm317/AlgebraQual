\documentclass[../AlgebraQualSolutions.tex]{subfiles}

\begin{document}

\section{Nilpotent}

\begin{prob}{F14}{F14.RF3}
	Prove that the set $N$ of nilpotent elements of a commutative ring $R$ is an ideal of $R$ and that $R/N$ has no nonzero nilpotent elements.
\end{prob}

\begin{proof}
	Suppose that $x,y \in N$. Let $m,n \in \N$ be such that $x^m = 0$ and $y^n = 0$. Then,
		\[(xy)^m = x^my^m = 0 \]
	meaning that $xy \in N$. Since $R$ is commutative, the Binomial Expansion Theorem holds and so
		\[(x + y)^{m+n} = \sum_{k=0}^{m+n} {{m+n}\choose{k}}x^ky^{m+n - k}.\]
	Observe that whenever $k \geq m$, $x^k = 0$ and whenever $k \geq n$, $y^k = 0$. When $0 \leq k \leq m$, $m + n - k \geq n$ and therefore, ${{m+n}\choose{k}}x^ky^{m+n - k} = 0$. When $m \leq k \leq m+n$, ${{m+n}\choose{k}}x^ky^{m+n - k} = 0$. Therefore, $(x+y)^{m+n} = 0$ meaning that $x + y \in N$.\\

	Assume now that $r \in R$ is arbitrary. Then,
		\[(rx)^m = r^mx^m = 0\]
	and so $rx \in N$. As $R$ is a commutative ring, this proves that $N$ is an ideal of $R$.
\end{proof}

\begin{proof}
	Suppose that $r + N$ is nilpotent in $R/N$. Choose $m \in \N$ such that $(r + N)^m = N$. For this $m$, it follows that $r^m + N = N$ or equivalently, $r^m \in N$. Since $r^m \in N$, there exists $n \in \N$ where $(r^m)^n = 0$. But this means that $r^{mn} = 0$ and so $r \in N$. Thus, any nilpotent element of $R/N$ is zero.
\end{proof}

\begin{prob}{F12}{F12.RF3}
\begin{enumerate}[(a)]
\item Prove that $\Z/m\Z$ has no non-zero nilpotent elements if and only if $m$ has no multiple prime factor.
\item Prove that every element of $\Z/m\Z$ is either nilpotent or a unit whenever $m$ is a prime power.
\item Prove that if $r$ is a nilpotent element of a ring with unity then $1-r$ is a unit.
\end{enumerate}
\end{prob}

\begin{proof}
    We show that $\Z/m\Z$ has some non-zero nilpotent element if and only if $m$ has some multiple prime factor. Suppose first that $x$ is a non-zero nilpotent element of $\Z/m\Z$. Then there exists an integer $k \geq 2$ such that $x^k = 0$. That is, $x^k$ divides $m$. Note that $x \neq 1$ because any power of 1 is 1. Let $p$ be any prime factor of $x$. Then, $p^k$ divides $x^k$ and therefore $p^k$ divides $m$. By assumption $k \geq 2$ and thus $m$ has a multiple prime factor.\\

    Conversely, assume that $m$ has some multiple prime factor. That is, there exists a prime $p$ such that $p^k$ divides $m$ with $k \geq 2$. Note that $p \in \Z/m\Z$ and since $p^k$ divides $m$, $p^k = 0$ in $\Z/m\Z$. Therefore $p$ is a nonzero nilpotent element of $\Z/m\Z$.
\end{proof}

\begin{proof}
    Let $p$ be prime and suppose that $m = p^N$ for some $N \in \N$. Let $x \in \Z/m\Z$ and suppose first that $p$ divides $x$. Then $x = pq$ for some $q$. Observe that
        \[x^N = (pq)^N = p^Nq^N = 0\]
    since $p^N = m \equiv 0$ in $\Z/m\Z$. That is, whenever $p$ divides $x$, it follows that $x$ is nilpotent. Suppose now that $p$ does not divide $x$. Because the only divisors of $m$ are powers of $p$, the greatest common divisor of $m$ and $x$ is 1. Therefore there exist integers $s,t$ where $1 = xs + mt$. As $mt = 0$ in $\Z/m\Z$, $xs = 1$ implying that $x$ is a unit.
\end{proof}

\begin{proof}
    Assume that $r \in R$ is a nilpotent element of a ring with unity. Then there exists $n \in \N$ such that $r^n = 0$. Then,
        \[1 = 1 - r^n = (1 + r + r^2 + \cdots + r^{n-1})(1-r)\]
    meaning that $\sum_{i=0}^{n-1}r^i$ is the multiplicative inverse of $(1-r)$. That is, $1-r$ is a unit.
\end{proof}

\end{document}