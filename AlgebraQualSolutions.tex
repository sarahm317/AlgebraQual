\documentclass{article}

\usepackage{titlesec}
\newcommand{\sectionbreak}{\clearpage}

\usepackage[margin=1in]{geometry} 
\usepackage{amsmath,amsthm,amssymb, graphicx, multicol, array}
\usepackage{enumerate}
\usepackage{mathrsfs,xcolor} 
\usepackage[shortlabels]{enumitem}
\usepackage[utf8]{inputenc}
\usepackage{tcolorbox}
\tcbuselibrary{theorems}
\usepackage{mathtools}
\usepackage{dsfont}
\usepackage{graphicx}
\usepackage{float}
\usepackage{esint}
\setlength{\parindent}{0pt} % Don't indent new paragraphs
\setlength{\headheight}{24pt} 

\DeclareMathOperator{\inter}{int}


% To allow for referencing
\usepackage{nameref}
\usepackage{hyperref}

% For separate files
\usepackage{subfiles}


%%%%%%%%%%%%%%%%%%%%%%%%%%%%%%%%%%%%%%
% Commonly used commands shortcuts
\newcommand{\N}{\mathbb{N}}
\newcommand{\Z}{\mathbb{Z}}
\newcommand{\R}{\mathbb{R}}
\newcommand{\Q}{\mathbb{Q}}
\newcommand{\C}{\mathbb{C}}
\newcommand{\F}{\mathbb{F}}
\newcommand{\inv}{^{-1}}
\newcommand{\Aut}{\textrm{Aut}}
\newcommand{\Inn}{\textrm{Inn}}
\newcommand{\Int}{\textrm{int}}
\newcommand{\ph}{\varphi}
\newcommand{\epsi}{\epsilon}
\newcommand{\incl}{\xhookrightarrow{}}
\newcommand{\ident}{\mathds{1}}
\newcommand{\sq}{\subseteq}
\newcommand{\scr}{\mathscr}
\newcommand{\lcm}{\textrm{lcm}}
\newcommand{\normal}{\triangleleft}
\newcommand{\range}{\mathrm{range}}
\newcommand{\im}{\mathrm{im}}
%%%%%%%%%%%%%%%%%%%%%%%%%%%%%%%%%%%%%%

% Template Information
\newcommand{\class}{Algebra Qualifying Exam}

% Make header with name and date etc.
\usepackage{fancyhdr}
\lhead{Algebra Qualifying Exam}
\pagestyle{fancy}

%%%%%%%%%%%%%%%%%%%%%%%%%%%%%%%%%%%%%%
% Creating solution environment
\newenvironment{solution}
  {\renewcommand\qedsymbol{}\begin{proof}[Solution]}
  {\end{proof}}
%%%%%%%%%%%%%%%%%%%%%%%%%%%%%%%%%%%%%%

%%%%%%%%%%%%%%%%%%%%%%%%%%%%%%%%%%%%%%
% Customize theorem, lemma, and definition environments
\theoremstyle{definition}
\newtheorem{defn}{Definition}[section]

\newtcbtheorem[use counter*=defn]{thm}{Theorem}%
{fonttitle=\bfseries}{thm}

\newtcbtheorem[use counter*=defn]{lem}{Lemma}%
{fonttitle=\bfseries}{lem}

\newtcbtheorem[use counter*=defn]{cor}{Corollary}%
{fonttitle=\bfseries}{cor}

\newtcbtheorem[use counter*=defn]{prob}{Problem}%
{fonttitle=\bfseries}{prob}

\newtcbtheorem[no counter]{LA}{Linear Algebra Done Right}%
{fonttitle=\bfseries}{prob}

%%%%%%%%%%%%%%%%%%%%%%%%%%%%%%%%%%%%%%

\begin{document}

\tableofcontents

\section{Directory}

\begin{itemize}
\item \textbf{Fall 2021}: 	G1 (Problem \ref{prob:F21.G1}), 
							G2 (Problem \ref{prob:F21.G2}), 
							G3 (Problem \ref{prob:F21.G3}),
							RF1 (Problem \ref{prob:F21.RF1}),
							RF2 (Problem \ref{prob:F21.RF2}),
							RF3 (Problem \ref{prob:F21.RF3}),
							LA1 (Problem \ref{prob:F21.LA1}),
							LA2 (Problem \ref{prob:F21.LA2}),
							LA3 (Problem \ref{prob:F21.LA3})
\item \textbf{Fall 2020}:  	G1 (Problem \ref{prob:F20.G1}), 
							G2 (Problem \ref{prob:F20.G2}), 
							G3 (Problem \ref{prob:F20.G3}),
							RF1 (Problem \ref{prob:F20.RF1}),
							RF2 (Problem \ref{prob:F20.RF2}),
							RF3 (Problem \ref{prob:F20.RF3}),
							LA1 (Problem \ref{prob:F20.LA1}),
							LA2 (Problem \ref{prob:F20.LA2}),
							LA3 (Problem \ref{prob:F20.LA3})
\item \textbf{Fall 2019}: RF1 (Problem \ref{prob:F19.RF1})
\item \textbf{Fall 2018}: G1 (Problem \ref{prob:F18.G1})

\end{itemize}

\subfile{Sections/PropNormSG}

\subfile{Sections/Sylow}

\subfile{Sections/GZGCyclic}

\subfile{Sections/OrbStab}

\subfile{Sections/Ideals}

\subfile{Sections/LADR}

\subfile{Sections/Unfinished}

\end{document}
