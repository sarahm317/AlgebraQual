\documentclass{article}

\usepackage{titlesec}
\newcommand{\sectionbreak}{\clearpage}

\usepackage[margin=1in]{geometry} 
\usepackage{amsmath,amsthm,amssymb, graphicx, multicol, array}
\usepackage{enumerate}
\usepackage{mathrsfs,xcolor} 
\usepackage[shortlabels]{enumitem}
\usepackage[utf8]{inputenc}
\usepackage{tcolorbox}
\tcbuselibrary{theorems}
\usepackage{mathtools}
\usepackage{dsfont}
\usepackage{graphicx}
\usepackage{float}
\usepackage{esint}
\setlength{\parindent}{0pt} % Don't indent new paragraphs
\setlength{\headheight}{24pt} 

\DeclareMathOperator{\inter}{int}


% For commutative diagrams
\usepackage{pst-node}
\usepackage{tikz-cd} 

% To allow for referencing
\usepackage{nameref}
\usepackage{hyperref}

% For separate files
\usepackage{subfiles}


%%%%%%%%%%%%%%%%%%%%%%%%%%%%%%%%%%%%%%
% Commonly used commands shortcuts
\newcommand{\N}{\mathbb{N}}
\newcommand{\Z}{\mathbb{Z}}
\newcommand{\R}{\mathbb{R}}
\newcommand{\Q}{\mathbb{Q}}
\newcommand{\C}{\mathbb{C}}
\newcommand{\F}{\mathbb{F}}
\newcommand{\inv}{^{-1}}
\newcommand{\Aut}{\textrm{Aut}}
\newcommand{\Inn}{\textrm{Inn}}
\newcommand{\Int}{\textrm{int}}
\newcommand{\ph}{\varphi}
\newcommand{\epsi}{\epsilon}
\newcommand{\incl}{\xhookrightarrow{}}
\newcommand{\ident}{\mathds{1}}
\newcommand{\sq}{\subseteq}
\newcommand{\scr}{\mathscr}
\newcommand{\lcm}{\textrm{lcm}}
\newcommand{\normal}{\triangleleft}
\newcommand{\range}{\mathrm{range}}
\newcommand{\Null}{\mathrm{null}}
\newcommand{\im}{\mathrm{im}}
\newcommand{\tr}{\mathrm{tr}}
\newcommand{\sm}{\setminus}
\newcommand{\ns}{\mathrm{null}}
\newcommand{\cyc}[1]{\left\langle #1 \right\rangle}
\newcommand{\ip}[2]{\langle #1, #2\rangle}
\newcommand{\norm}[1][\cdot]{\left|\left|#1\right|\right|}
\newcommand{\ol}[1]{\overline{#1}}
\renewcommand{\mod}[1]{\ \mathrm{mod}(#1)}
\newcommand{\rank}[1]{\rm{rank}(#1)}


\newcommand{\gap}{\centerline{\rule{\linewidth}{0.8pt}}}
%%%%%%%%%%%%%%%%%%%%%%%%%%%%%%%%%%%%%%
% For cycle notation

\ExplSyntaxOn
\NewDocumentCommand{\cycle}{ O{\;} m }
 {
  (
  \alec_cycle:nn { #1 } { #2 }
  )
 }

\seq_new:N \l_alec_cycle_seq
\cs_new_protected:Npn \alec_cycle:nn #1 #2
 {
  \seq_set_split:Nnn \l_alec_cycle_seq { , } { #2 }
  \seq_use:Nn \l_alec_cycle_seq { #1 }
 }
\ExplSyntaxOff

\newcommand{\perm}{\cycle[\quad]}

%%%%%%%%%%%%%%%%%%%%%%%%%%%%%%%%%%%%%%

% Template Information
\newcommand{\class}{Algebra Qualifying Exam}

% Make header with name and date etc.
\usepackage{fancyhdr}
\lhead{Algebra Qualifying Exam}
\pagestyle{fancy}

%%%%%%%%%%%%%%%%%%%%%%%%%%%%%%%%%%%%%%
% Creating solution environment
\newenvironment{solution}
  {\renewcommand\qedsymbol{}\begin{proof}[Solution]}
  {\end{proof}}
%%%%%%%%%%%%%%%%%%%%%%%%%%%%%%%%%%%%%%

%%%%%%%%%%%%%%%%%%%%%%%%%%%%%%%%%%%%%%
% Customize theorem, lemma, and definition environments
\theoremstyle{definition}
\newtheorem{defn}{Definition}[section]

\newtcbtheorem[use counter*=defn]{prop}{Proposition}%
{fonttitle=\bfseries}{thm}

\newtcbtheorem[use counter*=defn]{thm}{Theorem}%
{fonttitle=\bfseries}{thm}

\newtcbtheorem[use counter*=defn]{lem}{Lemma}%
{fonttitle=\bfseries}{lem}

\newtcbtheorem[use counter*=defn]{cor}{Corollary}%
{fonttitle=\bfseries}{cor}

\newtcbtheorem[use counter*=defn]{prob}{Problem}%
{fonttitle=\bfseries}{prob}

\newtcbtheorem[no counter]{LA}{Linear Algebra Done Right}%
{fonttitle=\bfseries}{prob}

%%%%%%%%%%%%%%%%%%%%%%%%%%%%%%%%%%%%%%
% Allow for claims inside a proof
\newlist{Claim}{description}{1}% allow 3 levels of nesting
\setlist[Claim]{labelindent=3em,leftmargin=*}

\let\originalqedsymbol\qedsymbol
\newenvironment{claim}{%
  % disable qed symbol if there is no star
  \let\qedsymbol\relax%
  \Claim\item[\textbf{Claim\csuse{theclaim}:}]%
  }{\endClaim\let\qedsymbol\originalqedsymbol}

%%%%%%%%%%%%%%%%%%%%%%%%%%%%%%%%%%%%%%

\begin{document}

\tableofcontents

\section{Introduction}

This document contains solutions to questions that have appeared on UCSB's Algebra Qualifying Exams. Some solutions are my own, some are inspired by Kyle Hansen and Zach Wagner's solutions, and others are inspired by Jacob Yakimov's solutions. Please email me with errors/corrections if you notice anything that should be changed.\\

The solutions are organized based on the topics addressed in the question. The problems statements also indicate which qualifying exams the problem has appeared. For ease of access, Section \ref{directory} lists the problems by qualifying exam.

\section{Directory}
\label{directory}

\begin{itemize}
	\item \textbf{Fall 2021}:
	\begin{itemize}
	\item G1 Problem \ref{prob:F21.G1} 
	\item G2 Problem \ref{prob:F21.G2}
	\item G3 Problem \ref{prob:F21.G3}
	\end{itemize}
	\item \textbf{Fall 2020}:
	\begin{itemize}
		\item G1 Problem \ref{prob:F20.G1} 
		\item G2 Problem \ref{prob:F20.G2}
		\item G3 Problem \ref{prob:F20.G3}
		\item RF1 Problem \ref{prob:F20.RF1}
		\item RF2 Problem \ref{prob:F20.RF2}
		\item RF3 Problem \ref{prob:F20.RF3}
		\item LA1 Problem \ref{prob:F20.LA1}
		\item LA2 Problem \ref{prob:F20.LA2}
		\item LA3 Problem \ref{prob:F20.LA3}	
	\end{itemize}
	\item \textbf{Spring 2020}:
	\begin{itemize}
		\item G3 Problem \ref{prob:S20.G3}
		\item RF3 Problem \ref{prob:S20.RF2}
		\item RF4 Problem \ref{prob:S20.RF4}
		\item LA3 Problem \ref{prob:S20.LA3}	
	\end{itemize}
	\item \textbf{Fall 2019}: RF1 Problem \ref{prob:F19.RF1}
	\item \textbf{Spring 2019}:
	\begin{itemize}
		\item LA2 Problem \ref{prob:S19.LA2}
		\item LA3 Problem \ref{prob:S19.LA3}
		\item LA4 Problem \ref{prob:S19.LA4}	
	\end{itemize}
	\item \textbf{Fall 2018}: 
	\begin{itemize}
		\item G1 Problem \ref{prob:F18.G1}
		\item G2 Problem \ref{prob:F18.G2}
		\item G3 Problem \ref{prob:F18.G3}
		\item RF1 Problem \ref{prob:F18.RF1}
	\end{itemize}
	\item \textbf{Spring 2018}:
	\begin{itemize}
		\item G1 Problem \ref{prob:S18.G1}
		\item G2 Problem \ref{prob:S18.G2}
		\item G3 Problem \ref{prob:S18.G3}
		\item RF1 Problem \ref{prob:S18.RF1}
		\item RF3 Problem \ref{prob:S11.RF3}
		\item LA1 Problem \ref{prob:S18.LA1}
		\item LA2 Problem \ref{prob:S18.LA2}
	\end{itemize}
	\item \textbf{Fall 2017}:
	\begin{itemize}
		\item G1 Problem \ref{prob:F17.G1}
		\item G2 Problem \ref{prob:F17.G2}
		\item G3 Problem \ref{prob:F17.G3}
		\item RF2 Problem \ref{prob:F17.RF2}
	\end{itemize}
	\item \textbf{Fall 2016}:
	\begin{itemize}
		\item G1 Problem \ref{prob:F16.G1}
		\item G2 Problem \ref{prob:F16.G2}
		\item G3 Problem \ref{prob:F19.G1}
		\item RF3 Problem \ref{prob:F16.RF3}
		\item LA1 Problem \ref{prob:F14.LA1}
	\end{itemize}
	\item \textbf{Spring 2016}:
	\begin{itemize}
		\item G1 Problem \ref{prob:S16.G1}
	\end{itemize}
	\item \textbf{Fall 2012}:	
	\begin{itemize}
	\item G1 Problem \ref{prob:F12.G1} 
	\item G2 Problem \ref{prob:F12.G2.a}, Problem \ref{prob:F12.G2.b}
	\item G3 Problem \ref{prob:F12.G3}
	\item RF1 Problem \ref{prob:F12.RF1}
	\item RF2 Problem \ref{prob:F12.RF2}
	\item RF3 Problem \ref{prob:F12.RF3}
	\item LA1 Problem \ref{prob:F12.LA1}
	\item LA2 Problem \ref{prob:F12.LA2}
	\item LA3 Problem \ref{prob:F12.LA3}	
	\end{itemize}	
	\item \textbf{Fall 2014}:  
	\begin{itemize}
	\item G1 Problem \ref{prob:F19.G1} 
	\item G2 Problem \ref{prob:S20.RF2} 
	\item G3 Problem \ref{prob:F14.G3}
	\item RF1 Problem \ref{prob:F14.RF1}
	\item RF2 Problem \ref{prob:F14.RF2}
	\item RF3 Problem \ref{prob:F14.RF3}
	\item LA1 Problem \ref{prob:F14.LA1}
	\item LA2 Problem \ref{prob:F14.LA2}
	\item LA3 Problem \ref{prob:F14.LA3}
	\end{itemize}
	\item \textbf{Fall 2013}:  
	\begin{itemize}
	\item G1 Problem \ref{prob:F13.G1} 
	\item G2 Problem \ref{prob:F13.G2} 
	\item G3 Problem \ref{prob:F13.G3}
	\item RF1 Problem \ref{prob:F13.RF1}
	\item RF2 Problem \ref{prob:F13.RF2}
	\item RF3 Problem \ref{prob:F13.RF3}
	\item LA1 Problem \ref{prob:F13.LA1}
	\item LA2 Problem \ref{prob:F13.LA2}
	\item LA3 Problem \ref{prob:F13.LA3}
	\end{itemize}

	\item \textbf{Spring 2013}:  
	\begin{itemize}
	\item G1 Problem \ref{prob:S13.G1} 
	\item G2 Problem \ref{prob:S13.G2} 
	\item G3 Problem \ref{prob:S13.G3}
	\item RF1 Problem \ref{prob:S13.RF1}
	\item RF2 Problem \ref{prob:S13.RF2}
	\item RF3 Problem \ref{prob:S13.RF3}
	\item LA1 Problem \ref{prob:S13.LA1}
	\item LA2 Problem \ref{prob:S13.LA2}
	\item LA3 Problem \ref{prob:S13.LA3}
	\end{itemize}
	\item \textbf{Spring 2012}:  
	\begin{itemize}
	\item G1 Problem \ref{prob:S12.G1} 
	\item G2 Problem \ref{prob:S12.G2} 
	\item G3 Problem \ref{prob:S12.G3}
	\item RF1 Problem \ref{prob:S12.RF1}
	\item RF2 Problem \ref{prob:S12.RF2}
	\item RF3 Problem \ref{prob:S12.RF3}
	\item LA1 Problem \ref{prob:S12.LA1}
	\item LA2 Problem \ref{prob:S12.LA2}
	\item LA3 Problem \ref{prob:S12.LA3}
	\end{itemize}
	\item \textbf{Spring 2011}:
	\begin{itemize}
		\item RF2 Problem \ref{prob:S11.RF2}
		\item RF3 Problem \ref{prob:S11.RF3}
	\end{itemize}
	\item \textbf{Fall 2003}:
	\begin{itemize}
		\item G1 Problem \ref{prob:F03.G1}
		\item G2 Problem \ref{prob:F03.G2}
		\item G3 Problem \ref{prob:F18.G3}
	\end{itemize}
\end{itemize}

\section{Results to Memorize}

	\subfile{Sections/CommonLA}

\section{Groups}

	\subfile{Sections/PropNormSG}

	\subfile{Sections/Groups}

	\subfile{Sections/Sylow}

	\subfile{Sections/Permutations}

	\subfile{Sections/GZGCyclic}

	\subfile{Sections/GroupActions}

	\subfile{Sections/Aut}

\section{Rings and Fields}
	\subfile{Sections/RFGen}

	\subfile{Sections/Ideals}

	\subfile{Sections/Nilpotent}

	\subfile{Sections/Polynomials}

	\subfile{Sections/FiniteFields}

	\subfile{Sections/ExtDeg}

	\subfile{Sections/GaloisGroup}

\section{Linear Algebra}
	\subfile{Sections/LinearMaps}

	\subfile{Sections/CharMinPoly}

	\subfile{Sections/InnerProducts}
	
	\subfile{Sections/CanonicalForms}

	%\subfile{Sections/LADR}

\subfile{Sections/Unfinished}

\end{document}
